\section{Genesis 2: Eine theologische Abhandlung}

\sectionmark{Gebein von meinem Gebein und \newline Fleisch von meinem Fleisch}

\begin{center}
\textit{\color{gray}Die Erschaffung der Frau in Genesis 2,18-25}
\end{center}

In den aktuellen Debatten der Gesellschaft spielt die Frage der Gerechtigkeit im Verhältnis der Geschlechter eine große Rolle. Oft wird dem Alten Testament vorgeworfen, die Frau abzuwerten und dabei wird nicht selten Gen 2,18-25 als Beispiel angeführt. Ob es dem Text gerecht wird, wollen wird uns nun anschauen.

\begin{multicols}{2}
\textbf{Text 18:}\\ \textit{Und der HERR, Gott, sprach: Es ist nicht gut, dass der Mensch allein ist; ich will ihm eine Hilfe machen, die ihm entspricht. 19 Und der HERR, Gott, bildete aus dem Erd- boden alle Tiere des Feldes und alle Vögel des Himmels, und er brachte sie zu dem Menschen, um zu sehen, wie er sie nennen würde; und ge- nau so, wie der Mensch sie, die lebenden We- sen, nennen würde, ⟨so⟩ sollte ihr Name sein. 20 Und der Mensch gab Namen allem Vieh und den Vögeln des Himmels und allen Tieren des Feldes. Aber für Adam [für ihn als Mensch] fand er keine Hilfe, ihm entsprechend. 21 Da ließ der HERR, Gott, einen tiefen Schlaf auf den Menschen fallen, sodass er einschlief. Und
 nahm eine von seinen Rippen und verschloss ihre Stelle mit Fleisch; 22 und der HERR, Gott, baute die Rippe, die er von dem Menschen ge- nommen hatte, zu einer Frau, und er brachte sie zum Menschen. 23 Da sagte der Mensch: Diese endlich ist Gebein von meinem Gebein und Fleisch von meinem Fleisch; diese soll Männin heißen, denn vom Mann ist sie genommen. 24 Darum wird ein Mann seinen Vater und seine Mutter verlassen und seiner Frau anhängen, und sie werden zu einem Fleisch werden. 25 Und sie waren beide nackt, der Mensch und seine Frau, und sie schämten sich nicht. (Elberfelder Bibel, Gen 2,18-25)}
\end{multicols}

Bevor wir uns der Analyse des Textes widmen, müssen wir einen kurzen Blick auf die Geschichte des Textes werfen. Die Entstehung des Pentateuchs (und des Buches Genesis als Teil des Pentateuchs) ist sehr komplex und vielschichtig. In der Forschung gibt es viele Modelle, aber es lassen sich drei Grundmodelle herausarbeiten: die Grundschriftshypothese/~Ergänzungshypothese, die Quellenhypothese/~Urkundenhypothese und die Erzählkranzhypothese/ Fragmentenhypothese. \\Diese Modelle existieren in verschiedenen Varianten und Kombinationen. Es lässt sich also nicht genau rekonstruieren, wie der Pentateuch entstanden ist. Ein paar Daten zur Entstehung des Pentateuchs sind jedoch in der aktuellen Forschung relativ unumstritten:

\begin{center}
\textit{\color{gray}\glqq Es gibt vorpriesterliche Textgruppen, die zum Teil aus dem Nordreich, zum Teil aus dem Südreich stammen. Diese Zyklen und Gesetzessammlungen wurden nach dem Untergang des Nordreiches (722) in Juda ediert und nach der Zerstörung Jerusalems von einer priesterlichen Gruppe miteinander verbunden. In der Perserzeit kamen viele weitere Texte hinzu (insbesondere das Buch Numeri) und im
4. Jh. Wurde dann nach Eingliederung des Buches Deuteronomium der Pentateuch als Tora Moses promulgiert.\grqq
}
\end{center}

Genesis 2,18-25 gehört zu den nicht-priesterlichen Texten. Es ist jedoch umstritten, ob es vor oder nach der Priesterschrift anzusetzen ist. Die zweite Schöpfungsgeschichte wurde oft als Abwertung der Frau – als Zweitgeschaffene – interpretiert. David J. A. Clines nennt folgende Argumente, um diese These zu stützen: Der Begriff Hilfe impliziert eine Unterlegenheit, eine Minderwertigkeit. Dass die Frau nichts macht, um dem Mann zu helfen, zeigt, dass sie nicht besonders wichtig ist. Der Mensch wurde von Anfang an als Mann geschaffen. In Gen 2,23 wird die Frau von dem Mann genannt. Dies sei ein Akt von Beherrschung und von Gott auch approbiert. Ob diese Argumente angemessen sind, wollen wir mithilfe des Textes überprüfen.

Der Begriff ézèr wird für eine Hilfe von Person zu Person, für eine das Leben rettende Hilfe, aber auch für eine militärische Hilfe verwendet. Hier erfährt der Geholfene einen Mangel und braucht eine Hilfe, eine Rettung. Wenn also von Über- und Unterordnung geredet werden sollte, wäre der Helfende (die Frau) dem Geholfenen (der Mann) überlegen. Es handelt sich jedoch hier um zwei Menschen aus einem Fleisch, zwei Menschen, die in Beziehung stehen. Es ist also nicht sinnvoll, von Über- und Unterordnung zu reden. Die Tatsache, dass die Frau von Gott selbst als Hilfe, Rettung, genannt wird, macht jeden Versuch, die Würde, die Identität und die Legitimität der Frau in Frage zu stellen, unmöglich.

Die Frau wird nicht zur Erfüllung irgendwelcher Zwecke dem Menschen gegeben. Sie wird ihm als Gefährtin zur Seite gestellt und er ihr als Gefährte. Sie ist primär die Gattin des Mannes, und er ist der Ehemann der Frau. Nur in einer zweiten Stufe wird die Frau zur Mutter. Sie wird dem Mann entsprechend erschaffen. Sie ist sein Pendant. Die Frau wird von Gott dem Menschen vorgestellt. Sie kommt nicht von sich selbst zu ihm, und er ruft sich auch nicht zu ihr. Der Mensch wacht auf und entdeckt die Frau. Dabei entdeckt er auch sich selbst als Mann, denn zuvor lag er noch vor der Unterscheidung der Geschlechter. Er ist nur Mensch. Er nimmt die Frau nicht; er bekommt sie. Der Mann handelt hier nur passiv; der eigentliche Akteur ist Gott.

Der Begriff âdâm bedeutet wortwörtlich Erdling. Er beschreibt in der Bibel sowohl die ganze Menschheit, als geschlechtliche Menschheit, als auch den Mann. Er wird generisch (Erdling, Menschheit), individuell (dieser Mensch) und auch als Eigenname verwendet. Es ist nicht immer klar, welche Bedeutung gemeint ist, denn sie ändert sich im Lauf der Geschichte. Manchmal sind auch mehrere Bedeutungen zugleich gedacht. In Gen 2 bezeichnet âdâm bis zur Erschaffung der Frau den asexuellen Menschen. Später bezeichnet er entweder die Gesamtheit der Menschheit oder den konkreten Menschen Adam.

Wenn wir uns V. 23 anschauen, lesen wir wortwörtlich: Für sie wird Männin gerufen. Der Autor spielt hier mit den Begriffen îsch und ishâh, die höchstwahrscheinlich etymologisch nicht verwandt sind. Es handelt sich hier nicht um eine präskriptive Aussage, sondern um eine deskriptive Aussage. Der Mann nennt nicht die Frau, wie er die Tiere genannt hat. Er verkündet seine Verwandtschaft mit ihr. Der Mann spricht hier zum ersten Mal, um seine paritätische Beziehung zur Frau zu verkünden. Er spricht nicht zu der Frau, sondern spricht vor Gott. Es handelt sich also nicht um einen Dialog, sondern um eine universale Aussage. Diese Aussage kann auch als Jubelruf verstanden werden. Nachdem die Tiere die Einsamkeit des Menschen nicht beenden konnten, kommt die Frau, und sie passt zu ihm: endlich eine passende Hilfe.

Für die Erschaffung der Frau wird das Verb bânâh (bauen) verwendet. Gen 2 spricht also von einer gebauten Frau. Bânâh hat drei Bedeutungen, die hier alle mitgemeint sind: ein Haus bauen (auch einen Tempel bauen), Nachfahren haben und glücklich sein. Gott ist der ideale Baumeister. Die Frau wird wortwörtlich aus der Seite des Menschen gebaut. Mann und Frau sollen also die zwei Seiten der Menschheit sein. Durch das Wegnehmen einer Rippe wurde der Mann verletzt und kann nur mithilfe der Frau gehen. Er ist also der Frau angewiesen.

Das Ein-Fleisch-Werden bezieht sich auf die Gründung einer Familie. Allerdings verlässt in altem Israel nicht der Mann seine Familie (matrilokal), sondern die Frau verlässt ihre Familie (patrilokal). Der Text steht also seiner patriarchalen Umwelt kritisch gegenüber: Die Frau muss nicht zum Mann zurückkommen, aus dessen Fleisch sie genommen worden ist, sondern sie ist zum Ziel der Lebenswege des Mannes geworden.

Wir können also die Argumente von David Clines zurückweisen und festhalten: Gen 2,18-25 verkündet keineswegs die Nachrangigkeit der Frau, sondern proklamiert die Gleichwertigkeit von Mann und Frau und beschreibt die Entstehung der zwei Geschlechter aus dem asexuellen Menschen. Die Sexualität wird hier auch relativ positiv bewertet. Mit der Erschaffung der Frau endete die schöpferische Tätigkeit Gottes. Sie ist sozusagen die Vollendung der Menschheit und der Schöpfung.


	\begin{flushright}
		\hfill\emph{Antoine Leyder Va!}
	\end{flushright}
	%	

%
%\begin{figurehere}
%	\begin{center}
%		\includegraphics[width=.8\linewidth]{Bilder/pios2}
%		\caption{Realer Aufbau} 
%	\end{center}
%\end{figurehere}