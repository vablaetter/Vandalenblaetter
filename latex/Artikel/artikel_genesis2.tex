\section{Gebein von meinem Gebein und Fleisch von meinem Fleisch}
\sectionmark{Genesis 2, eine theologische Abhandlung}


\vspace*{-3mm}
\color{gray}
\textit{Die Erschaffung der Frau in Genesis 2,18-25}
%\begin{figurehere}
%	\begin{center}
%		\includegraphics[width=\linewidth]{Bilder/andechs2.jpg}
%%		\caption{xxx} 
%	\end{center}
%\end{figurehere}


\begin{multicols}{2}

MARTINGSGANSESSEN


Als eine der herausragendsten Veranstaltungen im Wintersemester in unserer geliebten Vandalia findet das Martinsgansessen statt. Es bietet eine perfekte Gelegenheit, gemeinsam mit der Familie daran teilzunehmen, und das Wintersemester 2023/2024 bildet hier keine Ausnahme. Dank unseres AH Michael Oppitz, Richard Gurtner und der Organisation der Chargen konnten wir einen Abend mit leckerem Essen, Bundesbrüdern und Familie aus erster Hand genießen.

Der Tag begann frühmorgens mit der Essenszubereitung. Dank der Erfahrung unserer Alten Herren und der Unterstützung einiger Bundesbrüder war es kein großes Problem, alles für die Gänse vorzubereiten und sie für den Rest des Nachmittags im Ofen zu belassen, um den charakteristischen Geschmack zu erhalten, den wir später genießen würden. Als alles vorbereitet war, mussten wir selbstverständlich nur noch ein paar Bier öffnen, um auf die gute Arbeit anzustoßen. Mit der Ankunft weiterer AH und Familien wurde die Atmosphäre immer besser. Der Zeitpunkt für das Abendessen rückte näher, und so bestand die letzte Aufgabe der Aktiven Bundesbrüder darin, den Kneipsaal für alle unsere Gäste fertig einzurichten und zu dekorieren.

Um 19 Uhr trafen immer mehr bekannte Gesichter ein, die mit einem Lächeln und einem Bier alle Gäste begrüßten. Gleichzeitig konnte man Kinder im Vorraum und Kneipsaal herumlaufen sehen, während ihre Mütter sie ermahnten, vorsichtig zu sein, um sich nicht zu verletzen. Dies erfüllte Vandalia an diesem Abend mit Leben und verlieh uns allen eine schöne familiäre Atmosphäre. Schon beim Gang durch die Küche konnte man den Duft der fertigen Gänse riechen, der unseren Appetit anregte. Nun musste nur noch das Fass angeschlossen werden, das unseren Durst für den Rest des Abends stillen würde, und die Füchse begannen, die Gläser Bier und das Essen für diesen Abend zu servieren.

Mit etwas Verspätung begann das Abendessen, nachdem der Senior allen Gästen für ihr Erscheinen und allen Beteiligten für ihre Unterstützung bei der Zubereitung des Essens gedankt hatte. Der Abend wurde mit einem Gebet vor dem Essen eröffnet. Man konnte in den Augen aller Anwesenden die Begeisterung sehen, endlich das Essen zu probieren. Für diejenigen, die das Martinsgansessen zum ersten Mal erlebten, darunter auch mich, war es eine angenehme Überraschung, den köstlichen Geschmack der Gans zu probieren, die von unserem AH zubereitet wurde. Für diejenigen, die bereits in früheren Jahren dabei gewesen waren, war es eine Freude, in diesem Semester wieder dabei zu sein. Zwischen Bier und Gelächter war die gute Stimmung, die herrschte, spürbar. Auf der AH-Seite der Tische sah man, wie sie zusammen mit ihren Frauen und Kindern plauderten, während die Aktiven auf der anderen Seite miteinander scherzten und die Biergläser leerten.

Der Abend neigte sich dem Ende zu, und ich kann für alle sprechen, wenn ich sage, dass wir alle zufrieden und um eine schöne Erinnerung reicher waren. Während die Alten Herren langsam in ihre Häuser zurückkehrten, setzten die Aktiven die Nacht wie üblich noch ein paar Stunden fort. Am Ende dieser Nacht, während sich alle zur Ruhe begaben, blieb eine schöne Erinnerung zurück und der Wunsch, nächstes Jahr wiederzukommen, um erneut die köstliche Gans zu genießen, die es nur in unserer geliebten Vandalia gibt. Als Cosenior dieses Semesters kann ich sagen, dass es eine Veranstaltung war, deren Organisation ich mit großer Freude unterstützte, da ich glaube, dass gerade solche Abende besonders geschätzt werden, wenn man sich später an sie erinnert.
	
	%
	\begin{flushright}
		\hfill\emph{Derek Dominguez Va!}
	\end{flushright}
	%	
\end{multicols}
%

%
\begin{figurehere}
	\begin{center}
		\includegraphics[width=.8\linewidth]{./Bilder/1.1_martinsgansessen/DSC_2936.jpg}
		\caption{ein Consenior in seinem natürlichen Habitat} 
	\end{center}
\end{figurehere}
	

%
%\begin{figurehere}
%	\begin{center}
%		\includegraphics[width=.8\linewidth]{Bilder/pios2}
%		\caption{Realer Aufbau} 
%	\end{center}
%\end{figurehere}