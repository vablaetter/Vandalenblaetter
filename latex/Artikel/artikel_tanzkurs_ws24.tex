\section{Standardtanzkurs im November}
\sectionmark{Standardtanzkurs im November}

\begin{multicols}{2}
Anfang November fand seit langem endlich mal wieder ein Tanzkurs auf dem Vandalenhaus über zwei ganze Tage statt.  Das Ziel dieses war es, die ungeschickten und mit zwei linken Füßen gesegneten Bundesbrüdern fit für den Gaudeamusball im Januar zu machen. Auf dem Tagesprogramm standen die Standardtänze, vor allem aber der Cha Cha Cha und Langsamer Walzer.

Da tanzen durstig und hungrig (aber hauptsächlich durstig) macht, gab es an beiden Tagen einen ausgiebigen Brunch mit diveresen Leckereien und Sekt. 
Den ganzen Samstag über wurde viel getanzt, gelacht und bei manchen mehr und bei anderen weniger geratscht.
Der Ehrgeiz mancher Paare reichte sogar über die Dauer des Kurses hinaus, sodass nach dem Kurs noch fleißig weiter an Figuren gearbeitet wurde.

Im Anschluss gab es am Samstagabend das Martinsgansessen. Nach dem ausgiebigen Essen und ein paar Espresso-Martinis zum wach bleiben ging es für manche Paare noch mit dem Knotentanzen weiter, da dieser zwar keinen Platz im Tanzkurs gefunden hat, jedoch in das Repetoire eines jeden Verbindungsstudenten gehört.

Am zweiten Tag ging es dann wieder in der Früh mit den Standardtänzen weiter.
Der Tanzkurs endete mit einem ausgiebigen und langen Brunch.

Wir bedanken herzlich uns bei Gabriela Bog-Rampf und Joachim Rampf für den hervorragenden, witzigen und sehr schönen Tanzkurs.

	%
	\begin{flushright}
		\hfill\emph{Fabian Heinrich Va!}
	\end{flushright}
	%
\end{multicols}


\begin{center}
\begin{figurehere}\includegraphics[width=.7\linewidth]{./Bilder/1.6 Tanzkurs/Tanzkurs 4.png} 
\end{figurehere}
\end{center}	
	
