%% file: makros.tex
%% author: Thomas Satzger
%% date: 08.03.2007
%% purpose: Diplomarbeit, LaTeX-Definitionen


%% Theorem-Umgebungen:

%% Nummerierung nach sections, alle Theoreme mit dem gleichen Zaehler:
\theorembodyfont{\rmfamily}
\theoremstyle{break}% neue Zeile nach Theoremueberschrift
\newtheorem{defi}{Definition}[section]
\newtheorem{lemma}[defi]{Lemma}
\newtheorem{satz}[defi]{Satz}
\newtheorem{kor}[defi]{Korollar}
\newtheorem{bem}[defi]{Bemerkung}
\newtheorem{bemerkung}[defi]{Bemerkung}
\newtheorem{bemerkungen}[defi]{Bemerkungen}
\newtheorem{beispiel}[defi]{Beispiel}
\newtheorem{beispiele}[defi]{Beispiele}
\newtheorem{nothing}[defi]{\hspace{-1ex}}% Umgebung ohne Ueberschrift
%% TODO: nothing schaut furchtbar aus.


%% Umgebungen ohne Nummer:
% \input thbnone.sty
% \theoremstyle{bnone}% wie break, nur ohne Nummer
%\newtheorem{defi*}{Definition ohne Nummer}[section]
\newtheorem{beispiel*}{Beispiel}[section]
\newtheorem{beispiele*}{Beispiele}[section]
\newtheorem{bem*}[beispiel*]{Bemerkung}
\newtheorem{bemerkung*}[beispiel*]{Bemerkung}
\newtheorem{bemerkungen*}[beispiel*]{Bemerkungen}
\newtheorem{folgerung}[beispiel*]{Folgerung}
\newtheorem{folgerungen}[beispiel*]{Folgerungen}
%% Beweis:
%% qed-Kaestchen wird automatisch gesetzt, ausser es wurde schon 
%% von Hand (per \qed) gesetzt.
%% Ist eine Liste am Ende vom Beweis (d.h. itemize/enumerate/description),
%% dann muss \qed nach dem letzten Text (von Hand) eingefuegt werden.
\newtheorem{beweis}[beispiel*]{Beweis}
\let\tmpendbeweis=\endbeweis
\def\endbeweis{\smartqed\tmpendbeweis}


%% QED-Kasterl. Am sinnvollsten am Ende eine Absatzes
%% Klappt im Mathemodus nicht so toll.
\newif\ifQed\Qedfalse % speichern, ob manuell qed gesagt wurde 
%% (damit bei \end{beweis} nicht noch ein qed dazu kommt)
\newcommand{\qed}{\nolinebreak[4]\hfill\rule{1.2ex}{1.2ex}\global\Qedtrue}% 
%% qed, dass nur bei \Qedfalse gesetzt wird (fuer Beweisumgebung):
\newcommand{\smartqed}{\ifQed\relax\else\qed\fi\global\Qedfalse}


%% Keine Beweise, Beispiele in der Kurzversion:
%% TODO: Hier kann man noch optimieren (Beispiele rauslassen geht noch nicht).
\ifKurzversion
  \let\beweis=\comment
  \let\endbeweis=\endcomment
\fi


%% enumerate- und itemize-Umgebungen anpassen:
\partopsep=-\parskip% wirkt bei Absaetzen vor enumerate
\let\tmpenumerate\enumerate
\def\enumerate{\par%\vspace{-\topsep}
\vspace{-\parskip}%
\vspace{-\partopsep}%
\tmpenumerate\partopsep=0mm\itemsep=0mm\parsep=0mm}

%% Zwei neue Umgebungen f�r Listen: lista mit Buchstabenz�hlung, und listi mit r�mischen Zahlen
%% muss noch entwickelt werden
\newcommand{\listea}{\begin{enumerate}[\medspace (a)]}
\newcommand{\listeaend}{\end{enumerate}}
\newcommand{\listei}{\begin{enumerate}[\medspace $(i)$]}
\newcommand{\listeiend}{\end{enumerate}}
\newcommand{\liste}{\begin{enumerate}[\medspace 1.]}
\newcommand{\listeend}{\end{enumerate}}

\newcommand{\nobr}{\nolinebreak}

\newcommand{\vo}{ v\//\/o }


%Operatoren
%\DeclareMathOperator{\rang}{rang}
%\DeclareMathOperator{\sgn}{sgn}

% \vecn - Allgemeiner 1xn-Spaltenvektor mit Klammern ausnrum
% F�r Zeilenvektoren nimmt man besser \mat (siehe unten)
% Beispiel: \vecn{a \\ b \\ c}
%\newcommand{\vecn}[1]
%{\left(\begin{array}{c} #1 \end{array}\right)}

% \closure{TEXT} macht einen �berstrich �ber den TEXT,
% Denke an den Abschluss von Mengen
\newcommand{\closure}[1]
{\overline{#1}}

% \mat{SPALTEN}{INHALT} - Matrizen (mit normalen  ( ) )
% Beispiel: \mat{cc}{a & b \\ c & d}
\newcommand{\mat}[2]
{\left(\begin{array}{#1} #2 \end{array}\right)}

% \detmat{SPALTEN}{INHALT} - Determinanten-Matrizen ( | | statt ( ) )
% Syntax wie bei \mat
%\newcommand{\detmat}[2] % det ist leider vergeben
%{\left|\begin{array}{#1} #2 \end{array}\right|}

% \gmat{SPALTEN}{INHALT} - Matrizen mit geschweiften Klammern ( { } statt ( ) )
% Syntax wie bei \mat
\newcommand{\gmat}[2] % det ist leider vergeben
{\left\{\begin{array}{#1} #2 \end{array}\right\}}


%% Skalarprodukt:
%% Das Argument steht zwischen den Klammern.
\def\skalar#1{\ensuremath{\left\langle #1\right\rangle}}

%% Absolutbetrag (einfache senkrechte Striche, 
%% Groesse automatisch angepasst):
%% TODO: In chapter2.tex ab section 5 einfuegen.
\newcommand{\abs}[1]{\ensuremath{\left|#1\right|}}

%% Norm (doppelte senkrechte Striche, Groesse automatisch angepasst):
\def\norm#1{\ensuremath{\left\|#1\right\|}}

%% Erzeugendensystem einfach und doppelt
\def\erz#1{\ensuremath{\left\langle #1\right\rangle}}
\def\erzerz#1{\ensuremath{\left\langle\left\langle #1\right\rangle\right\rangle}}

%% Stacked Equal: = mit einem mathrm-Text dr�ber
\newcommand{\smequals}[1]
{\stackrel{\mbox{\tiny #1}}{=}}

\newcommand{\smany}[2]
{\stackrel{\mbox{\tiny #1}}{#2}}


% Personen-Namen werden in guten B�cher in Kapit�lchen geschrieben
\newcommand{\name}[1]{\textsc{#1}}


%% Mathe-Pfeile:
% Langer �quivalenzpfeil mit Abst�nden:
\def\aequiv{\ensuremath{\ \Longleftrightarrow\ }\xspace}
% Langer Implikationspfeil mit Abst�nden:
\def\impl{\ensuremath{\ \Longrightarrow\ }\xspace}
% Definitions�quivalenzpfeil mit Abst�nden:
\def\define{\ensuremath{\ :\Longleftrightarrow\ }\xspace}
% Beweis, Hin-Richtung und R�ck-Richtung:
% Verwendung: 
%\begin{description}\item[\hin]...\item[\rueck]...\end{description}
\def\hin{\glqq$\Longrightarrow$\grqq{}}
\def\rueck{\glqq$\Longleftarrow$\grqq{}}
% Geschweifte Klammer mit Kommentar unter Formeln:
% Zuerst den tiefgestellten Text, dann den normalen.
\def\underbraced#1#2{\ensuremath{\underset{\rule{0mm}{1.5ex}\makebox[0mm]{$\scriptstyle #1$}}{\underbrace{#2}}}}
% Transponiert:
%\def\trans{^{\mathsf T}}
\def\trans{^{\mathrm T}}

% Orthogonal
\def\ortho{^{\bot}}


%% TODO-Makro:
\def\todo#1{\ifDraft\par{\bfseries TODO: #1}\par\fi}
%% Marke zum Bild einfuegen, Argument ist z.B. Bildbeschreibung (caption):
\def\bild#1{\todo{Bild: #1}}


%% Makros spezifisch fuer Funktionalanalysis:
% Topologie (Schreibschrift-T):
\def\topo{\ensuremath{\mathcal T}}
% Bidual: Setzt zwei hochgestellte Sterne.
\def\bidual{\ensuremath{^{\ast\ast}}}

% URL einf�gen
\DeclareRobustCommand{\urlformat}[1]{\url{#1}} 

%% Bilder einfuegen:
%% Syntax:
%%   \abbildung{dateiname}[graphicx-optionen]{Beschriftung}[label]
%% Argumente: Dateiname (ohne Pfad und Endung), 
%% Optionen fuer includegraphics (optional), 
%% caption, Label (optional).
%% Beispiele:
%% \abbildung{hausdorff}{Zur Hausdorff-Eigenschaft}
%% \abbildung{hausdorff}{Zur Hausdorff-Eigenschaft}[fig:hausdorff]
%% \abbildung{hausdorff}[height=5cm,width=\textwidth,keepaspectratio=]{Zur Hausdorff-Eigenschaft}
\def\bilderpfad{./bilder/}
%\ifthenelse{\equal{\driver}{pdftex}}%
%  {\global\def\bildendung{.pdf}}%
%  {\global\def\bildendung{.ps}}% 
\makeatletter
\def\abbildung#1{\@ifnextchar[{\@bbildung{#1}}{\@bbildung{#1}[scale=1]}}
\def\@bbildung#1[#2]#3{%
  \@ifnextchar[{\@bbild@ngl{#1}[#2]{#3}}{\@bbild@ng{#1}[#2]{#3}}}
\def\@bbild@ng#1[#2]#3{%
  \begin{figure}[htb]
    \begin{center}
      \includegraphics[#2]{\bilderpfad#1}%\bildendung
      \caption{#3}
    \end{center}
  \end{figure}}
\def\@bbild@ngl#1[#2]#3[#4]{%
  \begin{figure}[htb]
    \begin{center}
      \includegraphics[#2]{\bilderpfad#1}%\bildendung
      \caption{#3}
      \label{#4}
    \end{center}
  \end{figure}}
\makeatother




\def\Input{\item[\textbf{Input: }]}
\def\Output{\item[\textbf{Output: }]}
\def\comment{\%}
\def\Given{\item[\textbf{Gegeben: }]}
\def\False{\ensuremath{\text{\bfseries False}}}
\def\True{\ensuremath{\text{\bfseries True}}}
\def\return{\ensuremath{\text{\bfseries Return}}}

\def\var#1{\ensuremath{\textit{#1}}}
%\renewcommand\algorithmiccomment[1]{ \{#1\}}%


\renewcommand\appendix{\par
\setcounter{section}{0}%
\setcounter{subsection}{0}%
\setcounter{figure}{0}
\renewcommand\thesection{\Alph{section}} %
\renewcommand\thefigure{\Alph{section}.\ arabic{figure}}} 

