\section{Rede des Seniors zur Abkneipe}
\sectionmark{Rede des Seniors zur Abkneipe}

\begin{multicols}{2}


Werte Kneipcorona,



das Semester neigt sich
wieder einmal dem Ende zu. Und damit einher geht auch mein Seniorat zuende. So
wie es mit vielen Dingen im Leben ist, kam das Seniorat schnell, zog sich dann
um den Kommers herum doch etwas in die Länge, aber zum Schluss ging es dann
ganz schnell. Rückblickend würde ich es mir daher gerne nochmal erlauben, das
letzte Semester etwas Revue passieren zu lassen.



Angefangen hatte alles mit
meiner Wahl als Senior, die ich selbstverständlich mit absoluter Mehrheit
gewann. Damit bin ich gleichauf mit der CDU auf Bundesebene und besser als alle
anderen Parteien. Wobei man natürlich anmerken muss, dass ich als einziger
angetreten bin. Naja kommen wir weiter zur Abkneipe. Die Senioratsrede damals
habe ich natürlich in bester Manier genauso wie diese Kneiprede erst eine
Stunde vor der Kneipe fertig geschrieben. Prokrastiniert wie ein Weltmeister
also; Aber
ich schweife ab.



Da war ich nun in der
Abkneipe des Sommersemesters 2024 während Benjamin mir die Senioratsbänder
umlegt. Ich muss ganz ehrlich sagen, damit fühlt man sich schwerer und das
nicht nur durch das Gewicht der zusätzlichen Bänder. Benjamin grinste mich an
und ich kann mir ganz genau denken was für eine Last mit dieser Übergabe von
ihm fiel.



Doch ein Seniorat oder
allgemein eine Hochcharge ist dann doch bereichernder, als es zuerst einmal
scheint. Es ist fast wie man für eine Klausur lernt um am Ende doch nicht
nachschreiben zu müssen.



Jedoch muss man hier nicht
mit unvorhergesehenen Konsequenzen rechnen. Das schlimmste, was einem passieren
kann, ist beim DC nicht dechargiert zu werden und die Charge nochmal machen zu
dürfen. Nebenbei angemerkt finde ich, dass wir diesen Absatz aus unserer
Satzung streichen sollten. Der Punkt ist, dass man durch eine Hochcharge etwas
wirklich mitgestalten kann und direkte Verantwortung lernt. Je größer das Fest,
desto größer natürlich auch die Verantwortung. Genau deshalb wünsche ich Daniel
im nächsten Jubelsemester alles Gute und bin mir sicher, dass es ein
unvergessliches Fest für 75 Jahre Vandalia sein wird.



Darauf:



Prost Corona!





	%
	\begin{flushright}
		\hfill\emph{Raphael Frank X}
	\end{flushright}
	%
\end{multicols}