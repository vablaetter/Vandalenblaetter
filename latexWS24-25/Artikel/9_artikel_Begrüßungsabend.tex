\section{Begrüßungsabend}
\sectionmark{Begrüßungsabend}

\begin{multicols}{2}
Am Freitag, dem 22. November, fand der Begrüßungsabend zum 120.
Gründungsfest am darauffolgenden Samstag statt, bei dem sich Damen, Freunde und
Gäste, geschätzte Alte Herren und deren Familien sowie Farben-, Cartell- und Bundesbrüder
in einer geselligen Runde zusammenfanden. Die Veranstaltung begann um 19 Uhr ct
im Festsaal der Verbindungsräumlichkeiten der K.D.St.V. Vandalia Prag zu
München im CV in der Friedrichstraße 34.

Nach der herzlichen Begrüßung aller durch den Organisator des Abends, den
damaligen Hohen Consenior Elia Strasser, und der Platzwahl begann der Abend mit
einer Vorspeise. Diesmal wurden die Teilnehmer des Abendessens von der Küche
mit Bruschetta und einem gemischten Salat verwöhnt. Während des Essens wurden
angeregte Gespräche geführt und Erinnerungen ausgetauscht. Nach einer schnellen
Erinnerung von Frau Chenu, auch ja die Immatrikulationsbescheinigungen
einzureichen, ging es mit guter Stimmung zum nächsten Gang.\\
Die Hauptspeise umfasste Reis mit Hühnerbrust auf Zitronen-Sahne-Sauce,
allerdings war der Reis zerkocht aufgrund von Problemen in der Küche. Berichten
zufolge hatte man dort mit der Sicherung zu kämpfen. Trotz dessen ließ sich
niemand die Laune verderben. Nach dem Hauptgang wurde ein Absacker in Form
eines Verdauungsschnapses von der Fuxia angeboten, welcher sehr gut ankam.\\
Die Atmosphäre war entspannt und herzlich, und man fühlte sich wohl. Die
Gespräche reichten von aktuellen Ereignissen bis hin zu der freudigen Erwartung
auf den morgigen Tag. Als runden Abschluss des Abends wurde als Nachtisch
Mousse au Chocolat mit roten Früchten serviert. Alle Gäste waren von der
gesamten Qualität der Speisen positiv überrascht, und es gab viel Lob für die
Köche.\\
Das ganze Drei-Gänge-Menü wurde von der Fuxia zu Tisch gebracht, und diese
versorgte auch alle mit Getränken, sodass es auch dahingehend nicht mangelte.
Außerdem bot sie Wein zum Verkauf an. Gegen 0 Uhr neigte sich der Abend dem
Ende zu, und alle gingen gesättigt und mit einem guten Gefühl im Bauch zu
Bette.

Insgesamt kann man nur noch allen Beteiligten und Mitwirkenden einen
herzlichen Dank für diesen wunderschönen, gemütlichen Abend aussprechen.

	%
	\begin{flushright}
		\hfill\emph{Jannis Jaschik Va!}
	\end{flushright}
	%	
\end{multicols}
