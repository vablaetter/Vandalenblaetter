\section{Rede des Seniors zur Antrittskneipe}
\sectionmark{Antrittskneipe}
Werte Kneipcorona,
\begin{multicols}{2}

Ein neues
Semester beginnt. Der aufmerksame Bundesbruder dürfte es bereits daran erkannt
haben, dass er sich auf einer Semesterantrittskneipe befindet. Andere haben
bereits die ersten Vorlesungen besucht. Und doch erkennt man es auch an etwas
anderem. Es ist nichts Konkretes wie ein Kalendereintrag oder ein Wecker, der
um 6 Uhr klingelt, nachdem ein heiterer Abend erst um 2 Uhr sein Ende fand. Und
doch kann man es anhand vieler Eindrücke spüren.

Zuallererst
finden nach dem meiner Meinung nach sehr gut gelungenen Ferialprogramm nun auch
wieder Semesterveranstaltungen statt – sei es der Antrittsgottesdienst der
Katholischen Hochschulgemeinde, der wirklich sehr gut besucht wurde, oder unser
kommender Begrüßungsabend nächste Woche. Dies sorgt natürlich dafür, dass man
wieder regelmäßig Bundesbrüder auf dem Haus sieht, nette Gespräche führen kann
und das ein oder andere leckere Bier nach einem längeren Uni-Tag gemeinsam
verzehrt.

In meiner
Rede bei unserer Abkneipe letztes Semester erzählte ich von diesem
unzufriedenen Gefühl, das man nach der Klausurenphase hat, obwohl man schon
alle Prüfungen geschrieben hat – und dass dieses Gefühl auch erst einmal
bleibt.

Jedoch bringt ein neues Semester auch immer ein ganz besonderes Gefühl mit sich. Jedes
Semester aufs Neue startet man mit einem unbeschriebenen Blatt, \newline einem Neuanfang
und für die meisten mit neuer Motivation und Energie. Denn auch wenn ein
Semester gefühlt immer sehr schnell vorbeigeht, unterschätzen die meisten, was
sie in einem Semester alles erreichen und schaffen können. Man arbeitet fürs
Studium, da bin ich mir bei den meisten sogar fast sicher, aber man arbeitet
auch an sich selbst. Man entwickelt sich weiter, knüpft neue Kontakte und ist
selbst nach nur einem Semester doch ein Stück weit ein anderer Mensch. Ich bin
mir sicher, dass auch unsere neuen Füxe diese Motivation und Energie in sich
tragen und wünsche ihnen einen guten Start ins Studium und in unser
Gemeinschaftsleben.

Dieses Semester ist bunt gefüllt mit mehreren Themenabenden wie dem Italienischen
Abend, der Bierauswahl oder dem Martinsgansessen. Wir fahren nach Tübingen auf
Aktivenfahrt und feiern unser \textbf{120. Gründungsfest am 23 November}, zu dem
ich euch alle sehr herzlich einlade. Auf einige interessante Vorträge folgen
der Musikalische Abend und die große Vandalen-Weihnachtsfeier. Es gibt also
vieles, auf das man sich freuen kann.

In diesem
Sinne freue ich mich auf alles, was unsere Verbindung auch in diesem Semester
blühen, wachsen und gedeihen lässt. Darauf ein Vivat, crescat,
floreat Vandalia ad multos annos!

Prost, Corona!

	%
	\begin{flushright}
		\hfill\emph{Raphael Frank Va! x}
	\end{flushright}
	%
\end{multicols}

