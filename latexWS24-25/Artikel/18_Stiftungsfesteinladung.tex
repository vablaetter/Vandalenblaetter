\section{Vorwort des antretenden Seniors zum Stiftungsfest}
\sectionmark{75 Jahre Vandalia in München}

Liebe Freunde der Vandalia,

Liebe Bundesbrüder,
\begin{multicols}{2}
Ein besonderes
Semester steht vor der Tür. Wir Vandalen feiern unser 120 jähriges Bestehen
sowie 75 Jahre Vandalia in München. Dazu laden wir euch ganz herzlich zu
unserem \textbf{Festwochenende vom 29 Mai bis 1 Juni} ein. Am Donnerstag, den
29 Mai beginnen wir mit einem lockeren Empfang bei uns auf dem Haus mit Sekt
und Häppchen, woraufhin am Freitag, den 30 Mai, der Höhepunkt, unser
Festkommers, steigt. Ganz besonders freut es mich den Festredner, den Erzbischof
von München und Freising, Reinhard Kardinal Marx anzukündigen. Am Samstag
werfen wir uns Vandalen dann in Schale, um bei Musik und Tanz, auf die MS
Starnberg auf dem Starnberger See zu steigen. Am Sonntag lassen wir unser Fest
mit einer Messe mit anschließendem Wirthausbesuch im Münchner Umland
ausklingen. Wie ihr seht, für Unterhaltung ist gesorgt. Daher schnell sein und auf
\textbf{vandalia75.de} Tickets ergattern.

Ableitend aus
unseren Jubiläum, steht das kommende Semester unter dem Motto \textit{„Patria“.} Das
wird auch durch unseren Semesterwahlspruch deutlich: \textit{„Heimat ist kein Ort,
Heimat ist ein Gefühl“ (Herbert Grönemeyer). }Wir werden uns auf Reise durch
unsere Heimatstadt München und deren historische Orte begehen. Untermahlt wird
das auch durch verschiedene Vortragsredner, die sich in den verschiedensten
Bereichen für unsere Heimat einsetzen. Ebenso wird, wie jedes Jahr, wieder das
traditionelle Maifest am 01. Mai stattfinden. Dazu natürlich herzliche
Einladung. Für nähere Details gerne ein Blick ins Semesterprogramm.

Mein letztes Wort aber
widme ich euch Philistern Vandaliae und eueren Familien. Leider ist über die letzten
Jahre, besonders in meiner aktiven Zeit, die Aktivität der Philister und daraus
resultierend die Anwesenheit der Familien leider etwas überschaubar. Vandalia
ist mehr als nur Fux, Bursch und Alter Herr, der brav seinen Altherrenbeitrag
bezahlt. Es ist ein gelebter Generationenvertrag (nicht nur finanziell), es ist
Familie. Daher meine
ausdrückliche Bitte: Kommt zum Maifest, kommt zum Stiftungsfest und das am besten
nicht allein. Bringt eure Frauen und Kinder mit. Lasst Vandalia wieder als Familie
erstrahlen, über den betagten Alten Herrn, die Ehefrauen bis hin zu den
Kindern. Danke!

In diesem Sinne auf
ein tolles Jubelsemester, was uns hoffentlich noch lange in \newline Erinnerung bleiben
wird!

Vivat, crescat, floreat Vandalia ad multos annos!

	%
	\begin{flushright}
		\hfill\emph{Daniel Rampf Va! x}
	\end{flushright}
	%
\end{multicols}

