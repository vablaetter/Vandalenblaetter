\section{Abschlusskneipe}
\sectionmark{Abschlusskneipe}

%\begin{figurehere}
%	\begin{center}
%		\includegraphics[width=\linewidth]{Bilder/andechs2.jpg}
%%		\caption{xxx} 
%	\end{center}
%\end{figurehere}



\hypertarget{Abschlusskneipe Wintersemester 2024/25}





\begin{multicols}{2}




Jedes gute Semester braucht einen guten Abschluss.
Nach einem erfolgreichen Semester, dank des Chargenkabinetts des Wintersemesters 2024/25, feierte unsere geliebte Vandalia die allseits bekannte Abkneipe,
bei der sich die Bundesbrüder vor der anstrengenden Klausurenphase ein letztes Mal im Semester zum Singen und Trinken versammeln.
Zusammen mit Cartellbrüdern und Farbenbrüdern begann die Abkneipe mit einer kleinen, aber eleganten Verzögerung von 20 Minuten, da immer mehr Besucher ankamen,
die kaum noch freie Plätze finden konnten.
Nach dem Einzug der Chargierten unter dem Kommando des hohen Seniors Rafael Frank begann die Abkneipe mit drei mutigen Schlägen.
Als das erste Lied von allen Teilnehmern gesungen wurde, merkte jeder, dass die Kneipe in großartiger Stimmung stattfinden würde.
Mit mehr als 30 Besuchern und über fünf anwesenden Verbindungen hatte man in jedem Colloquium die Gelegenheit, ein neues und lustiges Gespräch zu führen,
während die in diesem Semester neu rezipierten Füxe mit leeren Biergläsern hin und her liefen.
Später in der Kneipe erlebten auch zwei der Füxe ihren lang erwarteten Moment: die Burschung.
Es gibt für die Verbindung kaum einen bedeutenderen Moment, als wenn ein junger und motivierter Bursch den Reihen der Aktivitas beitritt und ihr neue Kraft verleiht.
Gratulor!! an unsere neuen Neoburschen: Benedikt Kirchhof und Felix Hantke.
Das zweite Highlight des Abends sollte im offiziellen Teil der Abkneipe stattfinden.
Ein hohes Chargenamt in der Verbindung zu übernehmen, ist nicht einfach.
Man trägt auf seinen Schultern die Last einer langen Geschichte und Tradition.
So zeigen wir als Mitglieder dieser Verbindung unsere Dankbarkeit für all die Unterstützung während des Studiums.
Doch die Verantwortung muss von den ohnehin schon müden Senior, Consenior und Fuxmajor dieses Semesters auf die nächsten Burschen in der Reihe übertragen werden – Burschen,
die bereit sind, die Herausforderung anzunehmen und unsere geliebte Vandalia zu vertreten.
Jeder aktuelle Charge wurde nach vorne ins Präsidium gerufen, wo er und sein Nachfolger die jeweilige Übergabe vornahmen.
Im kommenden Sommersemester habe ich die Ehre, die Charge des Fuxmajors zu übernehmen.
Als ich das Band des Fuxmajors von Magnus erhielt, erfüllte mich ein Pflichtgefühl gegenüber der Verbindung und den Füxen.
Eine Studentenverbindung kann nur durch engagierte Mitglieder den Lauf der Zeit überdauern und für weitere Generationen fortbestehen.
Genau deshalb werde ich mich mit ganzer Mühe meiner Aufgabe als neuer Fuxmajor widmen.
Zusammen mit dem Chargenkabinett des kommenden Sommer- und Jubiläumssemesters 2025 habe ich keinen Zweifel daran, dass es ein unvergessliches Semester für alle wird,
die das Glück haben, es zu erleben.



%\begin{itemize}
    %\item Leben auf selber Augenhöhe, ohne bösen Willen einander.
%\end{itemize} Für aufzählung mit stichpunkten

	\begin{flushright}
		\hfill\emph{Derek Dominguez Wk! Va!}
	\end{flushright}
	%	
\end{multicols}
%
%\begin{figurehere}
%	\begin{center}
%		\includegraphics[width=.8\linewidth]{Bilder/pios2}
%		\caption{Realer Aufbau} 
%	\end{center}
%\end{figurehere}