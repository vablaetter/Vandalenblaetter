\section{Vorwort des Philisterseniors}
\sectionmark{Vorwort des Philisterseniors}

%\begin{figurehere}
%	\begin{center}
%		\includegraphics[width=\linewidth]{Bilder/andechs2.jpg}
%%		\caption{xxx} 
%	\end{center}
%\end{figurehere}

\begin{multicols}{2}

Vor 75 Jahren wurde unsere Vandalia in München
wiederbegründet – fünf Jahre nach dem Ende des Zweiten Weltkriegs, in schweren
und herausfordernden Zeiten. Die Menschen in Europa und in Deutschland
versuchten jeden Tag, ihre Heimat wieder ein Stückchen mehr aufzubauen und in
ein normales Leben zu finden – jenseits von Krieg und Tod.\\
In dieser Zeit des Aufbaus haben sich auch einige Vandalen
in München getroffen und beschlossen, ihre Vandalia wieder zu begründen. Die
couleurstudentische Idee und ihre Verbundenheit zueinander und zur Verbindung
haben den Krieg überdauert und sie wieder „aktiv“ werden lassen. Herausragend!\\
Gott sei Dank können wir heute nur sehr bedingt
nachvollziehen, wie schwer und hart die Zeiten vor 75 Jahren waren. Aber gerade
deshalb muss und darf dieses Engagement einiger Bundesbrüder uns immer ein
Fingerzeig auf den Kern unseres Verbindungslebens und unserer Vandalia sein.
Ohne gegenseitige Freundschaft, Brüderlichkeit und Verbundenheit gibt es keine
couleurstudentische Verbindung.\\
Vandalia hat sich durch die Jahrzehnte hindurch behauptet,
weil immer wieder Bundesbrüder in der gemeinsamen Verbundenheit und
Freundschaft diese Flamme weitergetragen haben, die 1950 wieder angezündet
worden ist. Natürlich gab es auch in unserer Geschichte Höhen und Tiefen. Aber
die Brüderlichkeit und Freundschaft tragen uns auch im 21. Jahrhundert.\\
Herausfordernde Zeiten haben wir auch heute – nicht so sehr
bei der Frage, woher das Geld für den nächsten Monat kommt oder wie viel Essen
wir uns leisten können. Dafür sehen wir uns heute einem wachsenden Drang zur
Unverbindlichkeit und egozentrischen Selbstverwirklichung konfrontiert, die das
eigene Ego stets über die Gemeinschaft stellt. Verbunden mit einer schier
endlosen Anzahl an Angeboten für das eigene Konsumverhalten fällt es uns heute
unbeschreiblich schwer, Verbindlichkeit gegenüber anderen einzugehen. Unsere
Idee des Zusammenhalts in einer gemeinsamen Überzeugung bringt uns gegen den
Mainstream immer wieder an unsere Grenzen.\\
Nichtsdestotrotz spüren wir in den Räumen unserer Vandalia
immer den Auftrag aus der Überzeugung der Bundesbrüder von vor 75 Jahren. Dies
treibt uns an, Semester für Semester wieder miteinander zu feiern, zu
diskutieren, zu streiten und zu versöhnen, um unsere gemeinsame Überzeugung
weiterzuleben.

Es hat sich vor 75 Jahren gelohnt – es lohnt sich auch
heute!

Vor diesem Hintergrund freuen wir uns auf ein gemeinsames
Jubelsemester und auf tolle Veranstaltungen. Lasst euch anrühren und nehmt euch
die Zeit, wieder einmal das Vandalenhaus zu besuchen und mit uns zu feiern. Die
beste Gelegenheit dazu bietet das Stiftungsfestwochenende mit einem Kommers am
Freitag, den 30. Mai 2025, im Hofbräukeller mit dem Festredner Reinhard
Kardinal Marx, Erzbischof von München und Freising. Besonders freuen wir uns
auch auf die Schifffahrt auf dem Starnberger See am Samstag, den 31. Mai 2025.
Bei Speis und Trank, Musik und Tanz hoffen wir auf einen wunderschönen
gemeinsamen Abend. Am Sonntag werden wir nach einem gemeinsamen Gottesdienst
dieses Fest gemeinsam ausklingen lassen, um in Gottes Segen auch in die
nächsten Jahrzehnte gemeinsam aufzubrechen – immer mit einem „Vivat, crescat,
floreat ad multos annos Vandaliae“ auf den Lippen.

\end{multicols}
\begin{flushright}
		\hfill\emph{Dr. Michael Spickenreuther Va!}
	\end{flushright}