\section{Perspektivforum Vandalia - Klausurtagung}
\sectionmark{Perspektivforum Vandalia - Klausurtagung}


\begin{multicols}{2}	

Was heißt Zukunft? Dieser Frage gingen
unsere Vandalen am 26. Oktober 2024 auf dem Perspektivforum nach. Ein
Perspektivforum – das gab es noch nie; das erste seiner Art. Worum ging es? Die
großen Themen der Zukunft: Wie sich Vandalia weiterentwickeln will, welche
Rolle wir im Couleurwesen einnehmen, was mit der Deutschen Burse geschieht und
vieles mehr wurden in kleinem, freundschaftlichem Rahmen diskutiert. Mit dabei
waren vor allem Aktive, einige Inaktive und aus der Altherrenschaft Phil-xxx
Guido Herr sowie Bernhard Müller.

Bei einem Perspektivforum kommen keine
Beschlüsse heraus, sondern Vorschläge für Beschlüsse. Keine neuen Regeln,
sondern Reflexionen über Regeln. Es geschieht nichts Verbindliches, sondern
etwas Verbindendes: Man diskutiert gemeinsam, orientiert sich untereinander,
tauscht Ideen aus und nimmt sich die Zeit, die man auf einem Convent
normalerweise nicht hat. Man schaut aus der Vogelperspektive auf die Verbindung
– es geht um die nächsten Jahre und Jahrzehnte, nicht um das nächste Wochenende
oder das kommende Semester.

Ein Perspektivforum liefert, wie der Name
schon sagt, Perspektive: ein genaues Hinsehen auf den aktuellen Zustand und
einen Ausblick darauf, wo es hingehen soll. Eine Rückschau auf die Entwicklung
der Verbindung in den letzten Jahren und eine Vorschau auf die anstehenden
Herausforderungen.

Das Ganze findet zwanglos und in
entspanntem Rahmen statt. Zwischenzeitlich entwickelte sich in den hinteren
Reihen des Kneipsaals sogar eine Tee-Zeremonie. Dennoch haben das Diskutieren
Zeit und Energie gekostet. Sechs Stunden dauerte das Perspektivforum – am Ende
hat die Zeit fast nicht gereicht.

Was dabei herausgekommen ist, darf jeder
gerne beim nächsten Perspektivforum erfahren, wenn das Protokoll der letzten
Sitzung verlesen wird. Nur so viel sei verraten: Vor allem die Attraktivität
der Deutschen Burse, die Zimmervergabe im Wohnheim, die Ansehnlichkeit der
Vandalenräume, der Zustrom neuer Mitglieder, die Beziehung zwischen Aktivitas
und Altherrenschaft sowie das Selbst- und Außenbild der Vandalia waren Themen.
Man sieht: Reichlich Stoff zum Nachdenken – und sicherlich Themen, bei denen
der Input von „alten Hasen“ nicht schadet.

Dieser Artikel darf deshalb auch gerne als
Einladung gelesen werden: Wenn das Altherrendasein aus langer Erfahrung und
einem „Vogelblick“ auf das Studentenverbindungswesen besteht, dann ist Input
aus der Altherrenschaft sicherlich begehrt. Vielleicht freut sich das nächste
Perspektivforum also über Zustrom von Jung und Alt. Der nächste Termin ist
voraussichtlich im Wintersemester 2025/26.

	%
	\begin{flushright}
		\hfill\emph{Christian Schnurr Va!}
	\end{flushright}
	%	
\end{multicols}
%
