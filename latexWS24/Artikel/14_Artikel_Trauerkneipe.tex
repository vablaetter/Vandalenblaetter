\section{Trauerkneipe}

\sectionmark{Trauerkneipe}


Liebe Bundesbrüder,
\begin{multicols}{2}
Heute haben wir uns hier versammelt, um allen Bundesbrüdern zu gedenken,
die im letzten Jahr von uns gegangen sind.\\
Der recht bekannte Auszug aus der Bibel zeigt uns als katholische
Verbindungsstudenten, dass wir, egal wie finster die Zeit auch scheinen möge,
unserem Herrn vertrauen können, dass er ein nie erlöschendes Licht in unserem
Herzen ist. Dieser Halt hilft uns, durch schwere Zeiten zu gehen, denn wir
wissen, wir sind nie allein. Als Vandalia stehen wir nicht nur in Gemeinschaft
mit Gott, sondern sind auch in Gemeinschaft mit allen Vandalen im Lebensbund
verbunden. Dieser Lebensbund verbindet jeden von uns, vom frisch rezipierten
Fux bis zum hoch betagten Alten Herrn. Generationenübergreifend stehen wir für
gemeinsame Werte und Prinzipien. Diese prägen uns und machen uns zu dem, was
wir sind. Auch wenn wir mit jedem neuen Fuxen den Lebensbund in freudiger
Stimmung erweitern, so macht es doch auch jedes Mal aufs Neue betroffen, wenn
ein Bundesbruder aus diesem Lebensbund von uns geht.

Im vergangenen Jahr mussten wir die Bundesbrüder

Dipl. Kfm. (Diplom-Kaufmann) Karl Jähn v/o Kalle,

StD a.D. (Studiendirektor) Roland Kurz v/o Stunk und

Dr. med. (Doktor med.) Kurt Bröckner v/o Stachus

in die Hand Gottes geben.

Auch wenn wir als junge Vandalen oft keinen direkten Bezug zu den
Verstorbenen haben, so saß doch jeder von ihnen schon einmal an einem eurer
Plätze, ging auf Convente, Kneipen und Kommerse, brachte sich in der Verbindung
ein, war einer von uns.\\
Dieses Gleichbleiben der gemeinsamen Traditionen wird einem im Gespräch mit
jungen und älteren Philistern immer wieder klar.\\
Bei einigen von uns gegangenen Bundesbrüdern zeigen wir ein letztes Mal
unseren Respekt und so auch die Verbundenheit zur Vandalia. Manchmal geht diese
Verbundenheit über die Bundesbrüderlichkeit hinaus. So auch vor gut einem
halben Jahr, als wir an der Beerdigung von Ute Fest teilnahmen, die am 23. Juli
2024 verstorben ist. Auch sie stand unserer Verbindung nahe, genauso wie die
Bundesbrüder, die von uns gingen.

So nehmen wir heute Abschied von unseren verstorbenen Bundesbrüdern, die
für viele Vandalen gute Freunde auf ihrem Lebensweg waren.

\end{multicols}

	\begin{flushright}
		\hfill\emph{Raphael Frank Va! x}
	\end{flushright}
	%	
