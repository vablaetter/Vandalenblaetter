\section{Rede des Seniors zur Rezeptionskneipe}
\sectionmark{Rezeptionskneipe}
Hohe Kneipcorona,
\begin{multicols}{2}
Wie ihr bereits wisst, studiere ich Informatik. Dabei habe ich mit allerlei
technischen Spielereien zu tun, die von den normalen Benutzern heutiger
Anwendungen oft gar nicht bedacht werden. Dennoch wird der ein oder andere von euch mittlerweile von Künstlicher
Intelligenz gehört haben. Spätestens seit dem großen Durchbruch von ChatGPT
nutzen täglich Millionen Menschen große Sprachmodelle, um sich das Leben zu
erleichtern.

Jedoch fand Künstliche Intelligenz fast
direkt nach der Erfindung der ersten Computer Anwendung in vielen Bereichen –
sei es zur Erkennung von Schrift oder allgemeinen Mustern. Dennoch wird das
Thema erst seit ChatGPT, Gemini und Co. von der breiten Bevölkerung ernst
genommen. Natürlich möchte ich das Potenzial und die Fähigkeiten solcher
Technologien nicht abstreiten, dennoch erscheint ChatGPT für alle nur deshalb
so real, weil seine Hauptaufgabe darin besteht, möglichst natürlich die
menschliche Sprache zu imitieren. Denn Sprache ist das, was uns Menschen
ausmacht wie keine andere Kommunikationsform. Es ist das, was uns – zumindest
auf dieser komplexen Ebene – von anderen Tieren unterscheidet.

Und dennoch ist ChatGPT nicht die
Allzwecklösung, die viele heutzutage sehen. Egal, wohin man schaut, haben sich
zahlreiche Firmen den Hype zunutze gemacht und bewerben ihre Produkte, indem
sie „KI“ als Suffix anhängen. Genauso wenig, wie Kryptowährung bzw. Blockchain
unser komplettes Zahlungssystem in den nächsten Jahren revolutionieren wird,
wird Künstliche Intelligenz alle Berufe ersetzen oder die Lösung für alles
sein. KI und Chatbots haben ihre Anwendungen als Werkzeuge – und mehr nicht.
Denn die Mentalität, alle Probleme mit einer einfachen Patentlösung lösen zu
wollen, führt letztendlich nur in einen Kreislauf, der die eigenen Fähigkeiten
und den eigenen Horizont beschränkt.

Nur diejenigen, die mit Neugier und
Energie Weisheit suchen, ohne sich dabei auf Patentlösungen auszuruhen, werden
diese auch langfristig erhalten. Doch ich bin mir sicher, dass diese Neugier
bei jedem einzelnen Vandalen vorhanden ist, der sich Scientia als Prinzip zu
Herzen nimmt.

Gestärkt durch dieses Prinzip spreche ich
ein Vivat, crescat, floreat Vandalia ad multos annos!

Prost, Corona!

	%
	\begin{flushright}
		\hfill\emph{Raphael Frank Va! x}
	\end{flushright}
	%
\end{multicols}

