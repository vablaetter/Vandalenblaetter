\section{Wertschöpfung und Kreativität: Einblicke in das geistige Eigentum}
\sectionmark{Einblicke in das geistige Eigentum}

\begin{multicols}{2}
Geistiges Eigentum ist ein Thema, das oft
abstrakt wirkt – doch der Vortrag von Leo, einem auf dieses Gebiet
spezialisierten Juristen, bewies das Gegenteil. Unter dem Titel „Wertschöpfung
und Kreativität: Geistiges Eigentum und seine Bedeutung“ führte er sein
Publikum durch die komplexe, aber hochrelevante Welt der Urheberrechte,
Lizenzen und Schutzmechanismen für kreative Werke.

Besonders bemerkenswert war die Art der
Präsentation: Statt eines trockenen Monologs gestaltete Leo seinen Vortrag
interaktiv und ließ Raum für Fragen aus dem Publikum. Diese Möglichkeit, direkt
ins Gespräch zu treten, machte die Veranstaltung nicht nur lebendig, sondern
auch besonders lehrreich. Sein humorvoller Stil sorgte zudem für eine lockere
Atmosphäre – ein angenehmer Kontrast zu einem Thema, das oft mit Paragraphen
und juristischem Fachjargon assoziiert wird.

Als Kunststudent habe ich mit besonderem
Interesse zugehört, denn die Frage nach dem Schutz kreativer Arbeit betrifft
mich als Künstler unmittelbar. Welche Rechte habe ich an meinen Werken? Wie
kann ich sie wirtschaftlich nutzen, ohne ihre künstlerische Integrität zu
gefährden? Solche Fragen wurden praxisnah und verständlich beleuchtet.

Lieber Leo, abschließend bleibt mir nur,
mich für diesen bereichernden und inspirierenden Vortrag zu bedanken. Die
Mischung aus Fachwissen, Humor und Dialog hat gezeigt, dass geistiges Eigentum
alles andere als trocken ist und dass es sich gerade für Kunstschaffende lohnt,
sich intensiv damit auseinanderzusetzen (auch wenn die Anzahl der Vandalen
nicht allzu groß ist…).

Vielleicht bis bald auf einer Ausstellung für
zeitgenössische Kunst! :D
\end{multicols}

\begin{flushright}
		\hfill\emph{Elias Haindl Va!}
	\end{flushright}
	%	
			
