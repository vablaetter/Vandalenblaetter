\section{Brauereiführung im Airbräu}
\sectionmark{Brauereiführung im Airbräu}


\begin{multicols}{2}
Im Wintersemester 2024/25 fand am Sonntag, den 19. Januar 2025, ein Ausflug
an den Münch-ner Flughafen statt, der spirituell, kulinarisch und „bierisch“
geprägt war. Der Tag begann um 10 Uhr mit einem Besuch des Gottesdienstes in
der Christophorus-Kapelle am Flughafen. Die Pfarrer waren sichtlich überrascht
und erfreut, dass so viele Menschen – insbesondere junge – den Gottesdienst
besuchten.
\\
\\
\textbf{Die Führung im Airbräu}

Um 11:30 Uhr begann die Führung im Airbräu, der weltweit ersten
Flughafen-Brauerei. Der Führer erwies sich als äußerst kompetent und humorvoll,
was die Stimmung unter den Teilnehmern noch weiter auflockerte. Er beantwortete
geduldig alle Fragen und zeigte sich sogar flexibel, als die Führung etwas
länger als geplant dauerte – offenbar hatte auch er seinen Spaß daran. Als wir
ihn mit Fragen löcherten, blieb er kalt wie ein cooles Helles.\\
Während der Führung wurde die gesamte Anlage besichtigt, von den Sudkesseln
bis hin zur Lagerung des Biers. Besonders beeindruckend war der Moment, als wir
an den Malzkörnern riechen durften. Der intensive Geruch ließ kaum erahnen,
dass daraus später das köstliche Bier entsteht, für das das Airbräu bekannt
ist.\\
Für vor allem diejenigen, die im Januar auf Alkohol verzichteten, war die
kleine Zwischenmahlzeit während der Führung ein zusätzliches Highlight. Es gab
Brezen mit Obazda, Schweinebraten und Krautsalat – alles frisch zubereitet und
von regionalen Zutaten geprägt. Was mich persönlich am meisten dabei erfreut
hat, ist, dass es mengenmäßig sehr viel war, da kurzfristig „leider“ ein paar
Bundesbrüder abgesagt haben.\\
Zum Abschluss kehrten wir in der Wirtschaft des Airbräus ein. Leider ist
meinen beiden Conchargen Magnus und Raphael und mir ein Missgeschick
unterlaufen, da wir versehentlich vegane Burger bestellten. Der Burger erwies
sich, wie erwartet, als staubtrocken und war nur mit viel Soße genießbar.
Jedoch hat die positive Stimmung den Tag weiterhin geprägt und wir können auf
einen informativen und humorvollen Tag im Airbräu am Flughafen München
zurückblicken.

Mit bundesbrüderlichen Grüßen,

\begin{flushright}
		\hfill\emph{Elia Strasser Va! xx}
	\end{flushright}
		
\end{multicols}



