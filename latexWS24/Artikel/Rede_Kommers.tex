\section{Rede des Seniors zum Festkommers}
\sectionmark{Festommers}
Hohe Festcorona,
\begin{multicols}{2}
Da stand ich nun vor der Aufgabe, die,
glaube ich, jedem vergangenen oder zukünftigen Senior einen kalten Schauer über
den Rücken laufen lässt. Ich musste mir überlegen, was ich in meiner
Senioratsrede erzählen möchte. Mein erster Gedanke fiel dabei auf ein
klassisches Thema: die Politik. Denn auch, wenn ich mich allgemein als einen
politik-interessierten Bürger halte, muss ich mir schon eingestehen, dass ich
nicht mit weitreichendem Geschichtswissen glänzen kann und vielleicht gerade so
die Nationalversammlung 1848 in der Paulskirche zusammenbekomme. Ebenso
verstehe ich, dass Politik ein sehr schwieriges Thema ist. Fast hätte ich
gesagt, dass dies nur auf die heutige Zeit zutrifft, aber Politik hat die
Menschen schon immer polarisiert und in Rage gebracht.

Dennoch dachte ich mir, dass es möglich sein muss, frei über aktuelle
Geschehen zu reden und Entwicklungen zu bewerten. Denn für mich ist es genau
das, was aktuell viel zu kurz kommt: das Reden. Die Menschen reden zwar
miteinander und auch über Politik. Dennoch habe ich in den letzten Jahren das
Gefühl bekommen, dass sich die Menschen entweder gegenseitig nicht zuhören und
nur Wege finden wollen, den anderen schlecht darzustellen, oder in ihrer
Wahrnehmung der Welt zu sehr verzerrt sind, sodass jedwede Diskussion im Sande
verläuft. Natürlich spielen da moderne Medien wie TikTok, Instagram und Co.
eine maßgebliche Rolle. Man gibt den Medien nicht ohne Grund den Ruf, neben
Legislative, Judikative und Exekutive die vierte Macht in der Gewaltenteilung
Deutschlands oder allgemein der Welt zu sein.

Leider muss ich feststellen, dass sich durch Filterblasen und Algorithmen,
die einem nur das vorschlagen, was man selber sehen möchte, so gut wie jedes
Lager in der Politik isoliert immer weiter von anderen Meinungen, die nicht dem
eigenen Weltbild entsprechen, abgrenzt. Ein großes Problem, das ich dabei sehe,
ist die Möglichkeit, jeden Menschen auf der Welt zu erreichen. Versteht mich
nicht falsch, als Informatiker bin ich natürlich der letzte, der gegen eine
Globalisierung in der Informationsebene und den Ausbau des Internets als
solches ist. Das Internet hat die Welt sehr viel kleiner gemacht und Menschen
viel gegeben. Dennoch gibt es eben nicht nur ein Schwarz und ein Weiß.

Wurde früher ein, erlaubt mir diese Formulierung, Dorftrottel im ganzen
Dorf einfach nur ignoriert, hat sich dort im Gesamtbild der Menschen nicht viel
getan. Heute ist es jedoch nicht mehr nur ein Dorftrottel pro Dorf. Jeder
Mensch findet im Internet immer jemanden, der seiner Meinung ist, sei sie noch
so absurd. Füge dazu noch kaum bis gar nicht geprüfte Fakten in
Nachrichtenblogs hinzu, und es entsteht ein sich immer weiter aufheizender
Raum, in dem die polarisierendsten und lautesten Meinungen am meisten Anklang
finden.

Nun, die Politik ist keine Maschine, die
immer genau gleich funktioniert. Nein, sie ist vielmehr ein
Organismus, der sich dem Zahn der Zeit anpasst. Wer manche Entwicklungen der
letzten Wochen, Monate oder Jahre verfolgt hat, wird mich verstehen, wenn ich
sage, dass in der Politik alles möglich ist. Kaum wurde am Morgen des 6.
November das Ergebnis einer der wichtigsten Wahlen der Welt bekanntgegeben,
überschlagen sich am Abend die Ereignisse, und unsere Regierung löst sich auf.
Und ohne genauer wertend darauf einzugehen, fand ich die Art und Weise, wie
damit umgegangen wurde, nur mehr als repräsentativ für das, was ich gerade
vorgetragen habe. Von keiner Partei, sei es Regierung oder Opposition, wurde
respektvoll miteinander kommuniziert, um dieses Land handlungsfähig und stabil
zu machen.

Natürlich ist es nicht dumm, sich selbst gut darzustellen, sobald bekannt
wird, dass es bald Neuwahlen geben wird. Dennoch stehen dieses und viele
weitere Geschehnisse stellvertretend dafür, dass Deutschland momentan etwas
sehr Wichtiges fehlt: Verantwortungsvolle, rationale und rhetorisch gut
bewanderte Diskussionen über Politik. Und das ist etwas, woran jeder einzelne
Bürger meiner Meinung nach arbeiten sollte. Jeder sollte sich dazu berufen
fühlen, offen auf andere Menschen zuzugehen und bei sich gegenstehender Meinung
dennoch die Hand zu reichen und sich einander zu unterstützen.

Nach vielem Negativen möchte ich meine Rede jedoch mit etwas Positivem
beenden. Persönliche Erfahrungen meiner Gespräche mit Cartell- und
Bundesbrüdern und Abende wie dieser haben mir gezeigt, dass wir als
Verbindungen einen wichtigen Beitrag dazu leisten, eine offene und
intellektuell ehrbietende Ebene für gemeinsame Gespräche, Diskussionen und
gesellschaftlichen Austausch zu schaffen. Denn dies gibt mir Hoffnung,
verheißungsvoll in die Zukunft zu blicken, die uns alle betrifft.

In diesem Sinne: Vivant, crescant,
floreant Vandalia et Cartellverband ad multos annos!

Prost, Corona!

	%
	\begin{flushright}
		\hfill\emph{Raphael Frank Va! x}
	\end{flushright}
	%
\end{multicols}

