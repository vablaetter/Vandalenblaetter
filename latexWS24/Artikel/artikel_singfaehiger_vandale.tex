\section{Der singfähige Vandale}
\sectionmark{Der singfähige Vandale}

%\begin{figurehere}
%	\begin{center}
%		\includegraphics[width=\linewidth]{Bilder/andechs2.jpg}
%%		\caption{xxx} 
%	\end{center}
%\end{figurehere}

\begin{multicols}{2}
	
	Die Singfähigkeit des Vandalen zu garantieren ist seit jeher Aufgabe des Fuxmajors. In seinen Fuxenstunden klärt er nicht nur über die historischen Hintergründe der Verbindung auf, er bringt dem Fux auch das Vandalische Liedgut nahe. 
	Desweiteren steht es aber in der Verantwortung jedes Bundesbruders, auf den Kommersen das Liedgut weiterzugeben. Denn bei vielen der Kommers- und Kneipen-Schlager finden sich Eigenheiten, die nicht so im Notentext notiert sind. Ein Beispiel ist das furios-schnelle Ende bei “Ich schieß’ den Hirsch”. Auch die Zwischenrufe im “Münchner Lied” sind solche Eigenheiten, die man erst nach mehrmaligem Hören lernt und vielleicht sogar verändert (Da die Biergeschmäcker doch recht unterschiedlich sind.)
	
	Doch Singen ist mehr als nur das CV-Liederbuch auswendig zu kennen. In meinem Gesangsunterricht an der Musikhochschule arbeiten wir ständig an Intonation, Vokalfarbe, Interpretation, Phrasierung und Atmung, dem Fundament des Singens.
	Dies sprengt den Umfang des Machbaren für einen Hobby-Kneipensänger – oder tut es das? Einen Vorteil hat die feierliche Stimmung des studentischen Beisammenseins: die körperliche Aktivierung. Wie schon erwähnt, ist die Atmung das Fundament. Dazu gehört das Einatmen tief in den Bauch (Zwerchfell) und die Ausatemmuskulatur (seitliche Bauchmuskulatur), die den Klang verstärkt. Beim Lachen werden dieselben Regionen aktiviert, und jede gute Kneipe sollte hierzu genug Möglichkeiten schaffen.
	
	Der Vollständigkeit halber hier aber doch ein Tipp für den fleißigen Sänger. Nehmen wir nochmal den “Hirsch”. Wenn du das Wort “schieß” sprichst, was passiert mit deinen Mundwinkeln? Gehen sie intuitiv zur Seite, wie wenn du grinsen würdest?
	Beim Singen ist vor allem das Formen der Vokale wichtig, da diese den Klang beinhalten. Nimm also die Mundform des “sch”, die sogenannte Schnute (oder wie man es 2014 auf Instagram nannte: Duckface. Spreche nochmal \glqq schieß\grqq, ohne dabei die Schnute zu verändern. Merkst du den dunkleren Klang? Spreche es nochmal, spanne dabei deinen Bauch an und stell dir vor, wie ein dicker Opernsänger sprechen würde.
	
	Zum Abschluss darf ich darauf verweisen, dass das Singen auf keinen Fall zu akademisch werden soll. Vor allem in die Kneipe passt das, wie ich finde, nicht. Die Energie einer Kneipcorona habe ich in meiner Chorerfahrung bisher nur ganz selten genauso gespürt, auch beim Singen in den großen Konzerthäusern in Hamburg und Berlin zum Beispiel nicht immer.
	Deshalb darf ich getrost sagen, dass ich mir über die sängerische Zukunft unserer Vandalia keine Sorgen mache. 
	
	Wer sich gerne darüber hinaus an neuen Herausforderungen probieren möchte: Ich lade euch herzlich zum Chor der Prager Universitäts-Sängerschaft Barden ein, den ich seit zwei Jahren leite. Die Probe findet immer montags um 19.00 Uhr in der Leopoldstraße 255 statt.
	
	%
	\begin{flushright}
		\hfill\emph{Vincent Penschke Va! }
	\end{flushright}
	%	
\end{multicols}
%
%\begin{figurehere}
%	\begin{center}
%		\includegraphics[width=.8\linewidth]{Bilder/pios2}
%		\caption{Realer Aufbau} 
%	\end{center}
%\end{figurehere}