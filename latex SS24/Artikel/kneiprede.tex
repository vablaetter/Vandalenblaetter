\section{Kneipreden des WS 2023/24}
\sectionmark{Kneipreden des Wintersemesters 2023/24}

%\begin{figurehere}
%	\begin{center}
%		\includegraphics[width=\linewidth]{Bilder/andechs2.jpg}
%%		\caption{xxx} 
%	\end{center}
%\end{figurehere}

Liebe Bundesbrüder,

folgend habe ich Euch die Kneipreden aus dem vergangenen Semester zusammengeschrieben. Dies könnte insbesondere für die Bundesbrüder welche es leider nicht zu den Kneipen und dem Kommers geschafft haben, aber auch für diejenigen welche sie nochmal in Ruhe lesen möchten, interessant sein. Natürlich entsprechen die hier abgedruckten Reden nicht genau den auf der Kneipe Vorgetragenen. So habe zumindest ich einerseits bei diesen immer wieder mal auch spontane kleine Gedanken eingefügt. Und andererseits bin ich vor der Veröffentlichung hier nochmal über alle Texte drübergegangen, habe Rechtschreibung und Grammatik möglichst weit korrigiert und überschüssigen, langweiligen Ballast entfernt sowie Gedanken eindeutiger formuliert. Denn Intentionen und Aussagen, die man in einer Rede durch Gestik und Mimik unterstreichen kann, sind im Schrifttext schwerer zu fassen. So wünsche ich Euch viel Spaß beim Lesen und hoffe, dass die Texte Eure Gedanken anregen.
\\

\begin{multicols}{2}
\textbf{Rede des Seniors auf der Semesterantrittskneipe des Wintersemesters 2023/24:}

Werte Kneipcorona,
\\
im Jahre 1812 zogen gut eine halbe Millionen Mann über die Memel und bis in das zerstörte Moskau, bevor die Grande Armee Napoleons den russischen Weiten und Winter erlag und damit das Ende dessen Herrschaft über weite Teile der deutschen Lande bedeutete.
Vor ziemlich genau 210 Jahren, vom 16. bis 19. Oktober, besiegten die Alliierten in der Völkerschlacht von Leipzig die bis dahin unbesiegte französische Armee.
In Deutschland herrschte eine hoffnungsvolle Stimmung, eine Mischung aus vaterländischer Befreiung und Aufbruch aus der alten, durch absolutistische Monarchien geprägten Zeit.
Im Eindruck der Völkerschlacht schrieb der Dichter Max von Schenkendorf folgende Zeilen, welche Teil deutschen Liedguts wurden und auch im CV-Liederbuch abgedruckt, sind:
\\
\textit{"Freiheit die ich meine, die mein Herz erfüllt,
komm mit Deinem Scheine, süßes Engelsbild.
Magst du dich nicht zeigen, der bedrängten Welt?
Führest deinen Reigen nur am Sternenzelt?"}
\\
So lauten die ersten zwei Strophen. Die Zehnte dieses Gedichts ist der Wahlspruch des beginnenden Semesters:
\\
\textit{"Wo sich Männer finden, die für Ehr und Recht,
mutig sich verbinden, weilt ein frei Geschlecht.}"
\\
Auch wenn es mit über 130 Jahren noch deutlich länger gedauert hat, als jener Dichter und seinen Zeitgenossen hoffen, so haben wir heute inzwischen die von ihnen ersehnte Freiheit.
Sie wurde erst nach vielen weiteren Kriegen erreicht. Und unser Volk scheint auch nicht in der Lage, sie sich ohne Hilfe zu erkämpfen, sondern bedurfte der Befreiung von Außen.
Und auch wenn wir sie inzwischen haben, so zeigt es nicht nur die Geschichte, sondern auch die Gegenwart, dass zur Erhaltung der Freiheit diese auch stets verteidigt werden muss.
\\
Der für den Wahlspruch gewählte Strophe lassen sich grundlegend zwei Aussagen entziehen:
Einerseits:
\\ Es muss für \glqq Ehr und Recht\grqq ~eingestanden werden
und als zweites müssen sich hierzu \glqq Männer finden \grqq ~und \glqq mutig [...] verbinden \grqq ,
um Freiheit für das eigene Gemeinwesen zu erlangen.
\\
Die zweite Aussage ist hierbei deutlich einfacher zu verstehen und zu interpretieren, sodass ich hiermit anfangen möchte:
Es bedarf stets eines mutigen und gemeinsamen Einstehens. Gerade als Studentenverbindung sollte uns dies eigentlich klar sein und kaum weiterer Worte bedürfen.
Aber angesichts mancher Konflikte oder im Zuge einiger Diskusionen müssen wir uns auf diese für uns grundlegenden Dinge immer wieder besinnen und unser eigenes Handeln und unsere Worte anderen gegenüber und vor Allem über andere reflektieren. Denn, so sagt es dieser Halbsatz: \textit{Zerstrittene Einzelkämpfer erreichen ihre Ziele nicht}.
\\
Zwar gehört eine zivilisierte Diskussion und Streit um die richtigen Handlungen - gerne auch mit emotionalen Worten geführt - zum wichtigsten aller Güter. Insbesondere in einer so urdemokratischen Gruppe wie einer Studentenverbindung. Schließlich waren diese eine der wenigen Gemeinschaften, die jenes Prädikat zu jener Zeit schon verdienten.
Doch müssen wir dabei stets bedenken, am Ende wieder zusammen zu finden und Streitgräben und unschöne Emotionen gegeneinander zu überwinden.
Eben: Uns mutig zu verbinden.
\\
Etwas schwieriger finde ich es da, mit dem Ausdruck \glqq für Ehr und Recht\grqq ~umzugehen. Er ist nicht so geradewegs verständlich, wie das eben behandelte Verbinden. Aber eine Entdeckung vorneweg: Es wird ausgesagt, dass ein Zusammentun und gemeinsamer Einsatz zwar notwendig, aber eben nicht ausreichend sind, das \glqq frei Geschlecht\grqq ~zu sichern.
So stellen wir uns zunächst die Fragen: Für welche Ehr? Und für welches Recht?
Der erste der beiden Begriffe, die Ehre, habe ich unbeabsichtigt schon recht weit entwickelt. Denn als ich Euch soeben darüber erzählte, wie man sich verbinden sollte und was dabei zu beachten ist, da hab ich eigentlich das ehrhafte Verhalten, für welches man sich den Zeilen nach einsetzten muss, schon skizziert:
Ehrhaft bedeutet, dass man sich in die Augen sehen kann, in der aufrechten Überzeugung, einander gut zu behandeln, insbesondere wenn man dem Gegenüber nicht gerade freundlich gesonnen ist.
Für unsere Gemeinschaft, unseren Staat, bedeutet das, dass sie offen sein muss. Und dies meint nicht, dass jeder ihr zugehört, egal was er tut, woher er kommt oder wohin er geht. Offen bedeutet: Ich muss die Gewissheit haben, im Rahmen aller menschlicher Schwäche - sündigen tun wir leider alle -, dass ein jedes Mitglied der Gemeinschaft meine Entscheidungen respektiert. Was bei weitem nicht bedeutet, dass es sie gutheißen muss. Eine Gemeinschaft, in der dieser Respekt Teilen nicht entgegengebracht wird oder andersrum Teile ihn nicht entgegenbringen, kann auf Dauer ihre  Freiheit nicht erhalten.
Am besten lässt sich dies durch ein Gegenbeispiel belegen: Wenn wir in die, korrekt als Unrechtsstaat bezeichnete, ehemalige SBZ, pardon, die DDR, schauen, so ist diese untrennbar mit der Stasi und dem Spitzeln verbunden. Eine so im Geheimen durchgeführte Durchsetzung des Staats und auch der Bürger gegeneinander führte zu einem Misstrauensverhältnis zwischen Staat und Bürger, aber auch untereinander. Dies ist noch immer zu sehen, wenn man die Unterschiede des Vertrauens in Wahlen oder andere grundlegende Dinge des Gemeinwesens, anschaut.
Aber auch hier, selbst weit in die demokratischen Parteien wie CSU und SPD hinein, wird beispielsweise der Versuchung, den politischen Gegner zu dämonisieren und zu verleumden häufig nicht widerstanden. Gerade in den aktuellen Wahlkämpfen ist dies oft der Fall. Auf Dauer wird dies an dem Fundament unserer Zusammenlebens sägen.
So lässt sich zusammenfassen: In einer freiheitlichen Gesellschaft bedarf es stets eines ehrenvollen Umgangs, sprich mit offenem Visier, miteinander und des Einsatzes eines jeden Mitglieds hierfür.
Analog verhält es sich beim Recht, für welches wir uns dem Gedicht nach mutig verbinden sollten. Ohne ein gemeinsames Recht, grundlegende Regeln, welche für alle gelten und unabhängig des Status eines Mitglieds der Gemeinschaft durchgesetzt werden, ist eine freiheitliche Gesellschaft nicht möglich. Das Gegenteil von Recht ist die Tyrannei, wie sie in autokratischen Gesellschaften herrscht.
Aber auch bedeutende Gruppen in unseren Gesellschaften stellen das universelle - also für alle gleich geltende - Recht, ob Menschenrechte oder auch die engeren Bürgerrechte, immer häufiger in Frage. Ein Beispiel wäre hierbei die durchaus selektive Anwendung von Rechtsempfinden vieler Republikaner in den USA, aber auch mancher Vorschlag ehemaliger Bundesgesundheitsminister.
Und so sind Ehr und Recht als zwei eng verbundene Aspekte zu sehen. Beide müssen von uns stets hochgehalten werden und wir müssen sie leben. So wir dies tun, können wir uns sicher sein, weiterhin in einer Gesellschaft leben zu dürfen, welche uns unsere Freiheiten lässt und in welcher wir wachsen und gedeihen.
\\
Und was für den Staat im Großen gilt, das gilt auch in kleinerem Maßstab für unsere Verbindung. So mögen wir den Wahlspruch des Semester leben auf dass weiterhin gelte:

\columnbreak
Vivat, crescat, floreat Vandalia ad multos annos!
\\


\textbf{Ansprache des Seniors auf dem Gründungsfestkommers:}

Liebe Gäste und Damen, werte Cartell- und Bundesbrüder,
\\
bevor wir zur eigentlichen Rede kommen darf ich in einer kurzen Ansprache ein paar persönliche Gedanken mit Euch teilen.
\\
Als Katholiken stellt sich für uns eine Frage, welche wir ehrlich gesagt viel zu wenig stellen, vielleicht, weil wir die mit ihr eingehende Öffentlichkeit fürchten:
Was bedeutet die katholische Lehre für unser politisches Handeln? Wie sollen wir es ausrichten? Welchen politischen Leitlinien folgen wir?
\\
Häufig wird das Symbol des christlichen Glaubens auch für politische Zwecke missbraucht. Meist, weil man sich durch das Zurschaustellen dessen Sympathien oder eine vermeintliche Schärfung seines Profils erhofft, selten sogar weil es zur Hetze, man schaue etwa auf bestimmte Männer in weißen Roben in den USA, benutzt wird.
Doch davon abgesehen? Wie sollte unser politisches Handeln von unserem Glauben beeinflusst werden?
\\
Ein zentrales Gebot des Glaubens ist das Doppelgebot der Nächstenliebe: \textit{Liebe Gott und Deinen Nächsten wie Dich selbst}.
\\
Dieser zentrale Grundsatz, ebenso wie das Anerkennen, dass wir Sünder sind und das Vergeben derselben, ist meiner Meinung nach zu wenig in der Deutschen Politik beachtet. Denn: Auch angeblich katholische Cartellbrüder, im Vorstand zumindest namentlich christlicher Parteien, hetzen lieber gegen Menschen in Not. Und viel zu selten spielt in unserem wirtschaft- und sozialpolitischen Denken das Gebot der Nächstenliebe eine bedeutendere Rolle als die Besorgnis über geringe Unternehmensgewinne, die Furcht vor einer höheren Besteuerung von eigentlich luxuriösen Gütern oder ein Einfordern von Verantwortung bei denjenigen, welche über den Großteil des Vermögens unserer Gemeinschaft verfügen. Der christliche Glaube ist längst nur noch eine Minderheit unter den gesellschaftlich relevanten Einflüssen auf unsere Gesellschaft. Entkoppeln wir uns weiter von ihm, indem wir in unserem politischen Handeln statt an ihm an unseren eigensüchtigen - oder dafür wahrgenommenen - Anliegen ausrichten, sprechen wir ihm selbst auch noch jegliche Relevanz in der Gesellschaft ab. Sich dann noch katholisch zu nennen ist reine Selbsttäuschung.
Und so hoffe ich, dass die Nächstenliebe, wie wir sie untereinander praktizieren sich nicht in der Hilfe für den Bundesbruder erschöpft, sondern auch auf die erstreckt, welche uns physisch und emotional ferner sind. Wie es im heutigen Evangelium hieß: \glqq\textit{Was ihr für den Geringsten unter Euch getan habt, habt ihr für mich getan}\grqq.
\\


\textbf{Rede des scheidenden Seniors zum Hochoffiz der Semesterabkneipe:}

Liebe Bundesbrüder, werte Kneipcorona,
\\
im soeben verklungenen Lied haben wir gesungen:
\\
\glqq\textit{Wo sich Männer finden, die für Ehr und Recht
mutig sich verbinden, weilt ein frei Geschlecht}\grqq
\\
Dies haben wir als Wahlspruch für das vergange Semester gewählt und ich könnte jetzt eine Rede halten, in der ich diesen auslege und erkläre. Tun werde ich es aber nicht, da ich mich hierzu bereits an der Ankneipe weit ausgelassen habe.
Hingegen möchte ich mit einer Anekdote einleiten, welche sich auf jene Kneiperede begründet. So kam nach der Ankneipe ein Farbenbruder zu mir, sofern ich mich recht erinnere, aus dem BDIC, und fügte etwas zur Rede hinzu. Kurz gefasst ging es vor allem um die innere Ehre, welche doch das Wichtigste aller solcher Ehrgefühle sei: Also dass ich mit mir selbst im Reinen bin. Ein kurzes Beispiel, welches er mir ungefähr so darlegte: Ein großer König, dem für seine Macht von allen Ehrerbietung zuteil wird, ist doch viel ärmer an dieser wichtigen Ehre als das wohl extremste Beispiel Jesu Christi oder des Sünders am Kreuze, welche verspottet werden, aber in ihrem Innersten die Überzeugung haben können, mit sich und Gott im Reinen zu sein.
\\
Ausgehend von dieser Überlegung und einer Beobachtung vieler Festreden auf Kommersen und Kneipen bin ich auch zu der Auffassung gelangt, dass eine gute Rede nur auf zwei Ansätzen zu ihrem Thema aufbauen kann: Entweder, und ersteres ist der einfacherer Fall für den Redenschreiber, wählt man ein eher unverfängliches, neutrales Thema und erörtert dieses. Ein Beispiel wäre die bekannte Prinzipienrede oder das Erörtern eines Wahlspruches. Sicherlich ist dies eine meist unverfängliche und bei einer entsprechenden Thematik auch anregende Möglichkeit, häufig wird hieraus keine schlechte Rede. Der zweite Ansatz ist es, ein aktuelles - oft, aber nicht immer politisches - Thema als Grundlage zu wählen.
Und hierbei sind wir schon bei der inneren Ehre und meiner Beobachtung zu Reden auf Kneipen und Kommersen. Gerne wählt man hier ein Thema, welches für alle in der Corona und unseren ziemlich konservativen Kreisen unverfänglich ist. Wo alle zustimmen. Man könnte sich etwa über angebliche Schnitzelverbote oder ähnliches auslassen. Meiner Meinung nach ist dies keine gute Grundlage für eine hochoffizielle Kneip- oder gar Kommersrede. Bestenfalls kommt sie einem Stammtischgepolter gleich. Dort ist so etwas im Rahmen des allgemeinen sozialen Austausches angemessen, auf einer hochoffiziellen Feier einer akademischen Verbindung eher nicht. Und vor allem: Sie ist innerlich höchst unehrenhaft.
Stattdessen sollte man, wenn man solch ein Thema als Grundsatz wählt, die eigenen Positionen und das eigene Verhalten höchst kritisch beleuchten. So wäre es bei einer politischen Rede meiner Meinung nach eher angebracht, die Fehler der einem nahestehenden Parteien zu kritisieren. Beispielsweise könnte man sich in unseren Kreisen, welche historisch sehr eng mit der Union verwachsen sind, überlegen, ob nicht das C für Christlich oder das S für Sozial unter Betrachtung der Aussagen mancher Spitzenpolitiker heutzutage dieselbe Bedeutung wie das f in AfD oder das D in Demokratische Volksrepublik Korea haben. Denn das Hinterfragen eigener Standpunkte und denen der eigenen Umgebung ist vielmals effektiver als sich über weit entfernte Kreise lustig zu machen. Man lache besser über sich selbst als über andere. Und es ist der inneren Ehre, dem Mit-Sich-Im-Reinen-Sein somit viel dienlicher. Gemäß dem \glqq Wer frei von Sünde ist, werfe den ersten Stein\grqq ~sollte man in der Selbstkritik also einen Weg zum inneren Frieden sehen.
\\
Doch soweit zu der Einleitung und was mich dazu gebracht hat, die nun folgenden Betrachtungen über mich, uns und unsere Gesellschaft anzustellen.
\\
Denn, wenn ich mich einmal geistig ein paar Schritte aus dem Hier entferne, um aus etwas Distanz auf uns zu schauen, so fallen mir doch ein paar Punkte auf, welche entweder das Zusammenleben in der Verbindung stören und so unsere Zukunft gefährden können oder auf andere Weise unsere innere Ehre beschmutzen. So ist es einerseits so, dass wir uns häufiger bewusst werden sollten, was wir gerade von uns geben. Mag ein Spruch wie \glqq Alle an die Wand stellen!\grqq ~meinerseits zur Adelsbegeisterung eines Bundesbruders nur eine bewusste Überspitzung eines Ressertiments solcher Verehrung sein. Oder das Spielen von Melodien, zu welchen nicht nur auf ostdeutschen Volksfesten gerne \glqq Ausländer raus\grqq ~skandiert wird, noch mit einer lustigen Beinote geschehen. Es darf nicht passieren, dass man hierbei den schmalen Grad eines humorvollen Umgangs verlässt, was umso leichter passiert, desto näher einem diese Positionen sind. Doch wenn wir insbesondere bei Letzterem uns dann vor Augen führen, was manch ein Bundesbruder für Vorstellungen über die demographische Entwicklung in Europa hat oder auch nur ängstliche Vorurteile gegen so manchen politischen Gegner zeigt, so fällt mir unweigerlich auf, dass wir der Blasenbildung und damit verbundenen Entfernung von Teilen unserer demokratischen Gesellschaft - auch, aber nicht nur innerhalb der Verbindung - entgegenstehen müssen. Es fällt schwer, Freunden zu widersprechen, auch wenn man derselben Meinung ist, doch dieses Zurückholen auf einen kritischen Diskurs ist häufiger als es Getan wird nötig.
Soweit zu einer Kritik, was eigentlich nicht nur uns als Verbindung sondern die gesamte Gesellschaft betrifft.
\\
Hingegen betrifft uns Folgendes stärker:
Für eine enge Gemeinschaft wie die unsere ist ein freundlicher und respektvoller Umgang mit dem Bundesbruder unerlässlich. Bei der Aufnahme in den Lebensbund schwört man, allen Vandalen ein Freund und Bruder zu sein. Doch diesem Anspruch werden wir leider häufig nicht gerecht. Ob Alte Herren, welche sich nicht nur in immerhin ehrlicher Weise offen angehen, sondern auch im Gespräch mit anderen lästern. Oder die Aktiven, welche einem Bundesbruder ein Verhalten unterstellen, welches dieser nicht zu Tage legt, einfach weil sie gegenüber diesem Vorurteile haben und diese nicht zu überwinden bereit sind. Wenn Cartellbrüder eine Charge trotz schwerwiegender Verfehlungen mit Dank dechargieren, weil man mit demjenigen eben befreundet ist. Oder auch wenn die persönlichen Animositäten auch Wahlentscheidungen beeinflussen oder man deshalb eine Charge kritischer beurteilt, als es eigentlich gerechtfertigt ist.
All solche Vorkommnise sind ein Zukurzkommen bezüglich des Anspruches, welchen wir an uns stellen müssen. \glqq Steilheit\grqq ~,oder besser ehrenhaftes Verhalten, vor allem sich selbst gegenüber, endet nicht darinnen, die Kanne in zehn Sekunden zu leeren.
Denn nur solange die Bundesbrüder diesem Anspruch gerecht werden, kann der Bund wachsen und gedeihen. Nimmt andererseits ein fehlerhaftes Verhalten überhand so wird es langfristig zu einem Untergang desselben führen.
\\
Und so komm ich zu den letzten Punkt meiner kritischen Reflexion. Nicht erst beginnend, aber doch eindrücklich darin bestärkend, hat mich dazu eine Lesung über den Hohepriester und Samuel und die Worte unseres Bundesbruders David Theil über das dort nicht Vorgelesene in einem abendlichen Gottesdienst vor einigen Wochen. Die Quintessenz: Die Söhne des Hohenpriesters wollten oder konnten ihre Sünden und Fehler nicht sehen und wurden schließlich getötet womit diese Dynastie erlosch.
Denn immer häufiger denke ich darüber nach, dass wir - und damit meine ich konkret die selbsternannte \glqq Elite\grqq ~der westlichen Gesellschaften, für die wir uns wohl halten - eine der heuchlerischtsten Personengruppen sind. Ein- oder zweimal ist mir auch schon die Frage gekommen, ob nicht die Kämpfer der Hamas in ihrem Unwissen und ihrer durch die Gesellschaft, in welcher sie aufwachsen, verursachten Verblendung nicht eine größere Chance auf das ewige Leben hätten als wir. Bei denjenigen, die von vielen Korporierten als Geringverdiener diffamiert werden, bin ich mir dessen sogar recht sicher. Ich könnte jetzt beispielsweise fragen, wer aus der Corona an Heilig Abend in der Kirche war. Und es würden vermutlich 90\% der Hände gehoben. Wenn ich aber frage, wer abgesehen von den Verbindungsgottesdiensten auch nur viermal im vergangenen Jahr eine Messe besucht hat, wäre die Zahl sicher geringer oder würde wie in meinem Fall nur wenig drüber liegen. Und doch nennen wir uns eine katholische Verbindung - und dies, trotz all meiner Kritik auch zurecht, wie ich gerne ein andermal erläutere.
Doch davon abgesehen merke ich oft eine gewisse Arroganz in Verbindungskreisen. So geben andere Verbindungen aber auch Familien Monatsgehälter aus, um ein Wochenende lang der Schöpfung Gottes auf zwei dünnen Brettern den Buckel herunter zu rutschen. Wir merken in unserer Freiheit und unserem Konsum nicht, wie sehr wir dies auf Kosten zukünftiger Menschen und unter Ausbeutung größter Teile der Menschheit tun, den wir eben diese Freiheiten absprechen. Und andererseits stolzieren wir wie die Pfauen auf Veranstaltungen wie dem Gaudeamusball umher, trinken Sekt und wollen gerne Kasinoabende im feinem Smoking veranstalten.
Auch wenn ich die Gründe hier nur kurz skizziert habe, so drängt sich mir das Bild auf, dass wir wie die Hohepriester sind, die ihre Nase so hoch in den Wolken haben und sich unangreifbar wähnen, doch deren Untergang bevorsteht, spätestens vor dem jüngsten Gericht.
\\
Ich könnte mich gerade zu diesem Thema noch eine halbe Stunde auslassen, doch ist hiermit niemandem gedient.
Hingegen ist es mir wichtig, nun klarzustellen, wofür diese Rede gedacht ist. Denn die Einsicht, wo die eigenen Fehler liegen, das Bewusstwerden über die eigene Fehlbarkeit und der Versuch der Besserung sind die wichtigesten Schritte auf dem Weg zu ebendieser.
Und so finde ich es eine unendliche Hoffnung gebend zu wissen, dass Gott eben hierfür von uns hingerichtet wurde und für uns gestorben ist. Denn das Sündigen an den Anderen und vor allem sich Selbst ist mit dem menschlichen Wesen leider unabdingbar verknüpft. Und Gott hat uns eine Brücke über diese Gräben geschaffen. So können wir einander die Hand reichen, uns beim Tanz auf dem Ball unserer Privilegien bewusst werden, vielleicht mal das Statussymbol links liegen lassen und anderen Helfen. Und so bin ich, trotz oder gerade wegen der Möglichkeit zur Reflektion in der heutigen Rede guter Dinge für unsere Zukunft.
Denn wir werden die Fehler überwinden und die heutige Amtsübergabe im Zeichen einer langen Tradition solcher Übergaben zeigt:
\\
Wenn wir uns auf unsere Grundlagen im katholischen Glauben besinnen, wird ebenso wie wir selbst
\\
Vandalia wachsen, blühen und gedeihen!
\\
Und hierauf, auf diese Zuversicht möchte ich mit Euch allen anstoßen.
\\


	%
	\begin{flushright}
		\hfill\emph{Konrad Schönleber RBo! Va!}
	\end{flushright}
	%	
\end{multicols}
%
%\section*{Aus dem Semester}
\begin{figurehere}
	\begin{center}
		%\includegraphics[width=.45\linewidth]{Bilder/kneipe/01} %\hspace{5mm}
		
		%		\center{\caption{Die Fuxia bei der Arbeit}}
	\end{center}
\end{figurehere}