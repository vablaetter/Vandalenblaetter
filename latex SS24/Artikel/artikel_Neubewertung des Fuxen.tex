\section{Neubewertung des Fuxendaseins}
\sectionmark{Neubewertung des Fuxendaseins}

%\begin{figurehere}
%	\begin{center}
%		\includegraphics[width=\linewidth]{Bilder/andechs2.jpg}
%%		\caption{xxx} 
%	\end{center}
%\end{figurehere}

Was ist der Fux?

Dummfux. Kackfux. Fux: Stoff!

\begin{multicols}{2}
Es gibt nicht viel zu beneiden in manchen Momenten an der Art und Weise wie man als Fux behandelt wird. Man ist nicht da, mit den anderen auf gleicher Augenhöhe, sondern ist eine Stufe darunter, führt die niedrigen Arbeiten aus, wartet auf die Erlösung seiner Schicht.
Das Fuxendasein darf sich emanzipieren. Es sind nicht wir Burschen, die der Verbindung seinen Gehalt geben, sondern eine Verbindung wächst und gedeiht, verblüht und erkrankt am neuen Zustrom seiner Mitglieder. Der Fux ist kein Mitglied auf Probe. Er stellt die Verbindung auf die Probe. Es liegt an ihm, ob er entscheidet uns aufzuwerten oder durch seine Abkehr wieder in die Mittelmäßigkeit zu verstoßen. Für uns bleibt es nur zu warten, wie er sich entscheidet, es darf uns genügen den Rahmen zu bilden für die Entscheidung eines Menschen, der noch keine Abhängigkeit von uns erfährt.
Der Fux ist das zukünftig Kommende. Was die Verbindung in einigen Jahren ist, das liegt alles im Fux jetzt verborgen und wachsend. Es liegt an uns dem Fux zu dienen in der Weise, dass wir nicht großspurig ihm die Welt und das Verbindungsdasein erklären und verkünden. Es liegt an uns die Welt und das Verbindungsdasein in einer Weise zu strukturieren und zu verändern, die dem Fuxen seinen freien Lauf gewährt. Er bedarf der Freiheit sich umzublicken, unbedrängt die Räume zu erkunden und in seiner Rolle als der Neue von Repressalien frei zu bleiben.
Der Fux bedarf unseres höchsten Respekts. Nicht weil er neu ist, sondern weil er das ist, was die Verbindung sein wird. In ihm spiegelt sich auch das, was wir im Moment leisten. Die Qualität des Verbindungswesens hat sich schon immer an der Höflichkeit, Zwischenmenschlichkeit und dem Respekt gezeigt, mit dem wir einander auf Augenhöhe begegnen. Wer davon nichts weiß, dem ist die Idee der Brüderlichkeit zugunsten einer ungleichen Hierarchie verlorengegangen. In einem solchen Raum, der den Fux als Bundesbruder zweiter Klasse betrachtet, hat die Zukunft auch nichts verloren. Es gibt hier nichts zu holen für den Fuxen. Weshalb sich einer Gemeinschaft anschließen, die keinen Respekt für niedere Tätigkeiten hat? Weshalb der Unverschämtheit und Arroganz, die einem entgegengebracht wird, mit Offenheit und Zuversicht, dass es sich ändern würde, entgegentreten? Es gibt für den Fuxen nur etwas zu holen, wenn wir ein Raum sind und werden, in dem Hierarchie als etwas gezeichnet wird, das als Tätigkeitsverteilung, aber nicht als Klassenunterschied das Verbindungsleben strukturiert. Nicht jeder erfüllt dieselbe Aufgabe, aber jeder ist als Gleicher an der Verbindung beteiligt. Dass der Fux noch vor dem langfristigen Beitritt zur Verbindung steht, ist Aufgabe für uns, ihm zu zeigen, was es heißt, bereits jetzt einer von uns zu sein.
Ein Bruder zu sein, ist nicht einfach. Jeder mit Geschwistern weiß das. Sich das Wort Brüderlichkeit so zur Aufgabe zu machen, wie Verbindungen es tun, muss für uns heißen:
\begin{itemize}
    \item Leben auf selber Augenhöhe, ohne bösen Willen einander.
    \item Eingeständnis, dass wir dieselben Gerechtigkeitsansprüche haben, dasselbe Recht auf Wohlwollen und Respekt.
    \item Ermöglichung eines sinnvollen und erfüllenden Beitrags zum gemeinschaftlichen Leben. Hierarchie wo es nötig ist, aber nicht, wo es schadhaft ist.
\end{itemize}
Das, was ich heute als Verbindungsstudent bin, bin ich zum großen Teil geworden während der Fuxenzeit. Was mich an der Verbindung hält, stammt aus dieser Zeit. Was ich dem Neuen schulde, ist dieselbe Freiheit und Gleichheit, die mir an vielen Stellen entgegengekommen ist, und die mich bewogen hat, das Schlechte in Kauf zu nehmen. Wer die Verbindung ändern will, der bedarf des Fuxen.
	\begin{flushright}
		\hfill\emph{Christian Schnurr Va! }
	\end{flushright}
	%	
\end{multicols}
%
%\begin{figurehere}
%	\begin{center}
%		\includegraphics[width=.8\linewidth]{Bilder/pios2}
%		\caption{Realer Aufbau} 
%	\end{center}
%\end{figurehere}