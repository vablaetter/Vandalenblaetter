\section{Vorwort des Philisterseniors}
\sectionmark{Vorwort des Philisterseniors}

%\begin{figurehere}
%	\begin{center}
%		\includegraphics[width=\linewidth]{Bilder/andechs2.jpg}
%%		\caption{xxx} 
%	\end{center}
%\end{figurehere}

\begin{multicols}{2}

Die Jahreshauptversammlung des CV fand in diesem Jahr in Berlin statt. Sie war geprägt durch eine Diskussion über unser Prinzip religio. Der ein oder andere hat davon vielleicht in der ACADEMIA etwas gelesen. Die Diskussionen waren auch in den vorausgegangenen Regionaltagungen zum Teil sehr hitzig. Ohne die Details hier auszubreiten, möchte ich aber einen kurzen Blick darauf werfen. Die Tatsache, dass ein Vorort seine Amtszeit dadurch prägen will, etwas zu verändern und dann nicht bis zur CV damit wartet, sondern bereits zu Beginn seiner Amtszeit einen Diskurs startet, halte ich für richtig. Jedes Chargenkabinett - sei es im Vorort oder auch in der Verbindung - hat eine zeitlich begrenzte Amtszeit. Veränderungsprozesse am Ende der Amtszeit zu starten, sind zumeist sinnlos. Aus dieser Sicht betrachtet, war das Vorgehen des Vororts Berlin sehr gut geplant und kann und sollte ein Vorbild für zukünftige Chargenkabinette sein.
Über den Inhalt wird und wurde trefflich diskutiert und gestritten. Aus meiner Sicht gibt es hier durchaus gute Impulse. Über andere Thesen darf und muss man auch geteilter Meinung sein. Wichtig scheint mir aber, dass wir Themen und auch unsere Prinzipien diskutieren. Ohne Diskussion über unsere Prinzipien und wie wir sie auch in Zukunft leben, stellt sich jedem von uns über kurz oder lang die Frage nach der Bedeutsamkeit unseres Verbandes.
Gute Diskussionen über strittige Themen können verbinden, das Profil stärken und sollten uns als akademischen und katholischen Verband ausmachen. Als wichtige und beruhigende Botschaft aus diesen Diskussionen ist mir unser einigendes Motto immer wieder lebendig vor Augen geführt worden: \glqq In necessariis unitas, in dubiis libertas, in omnibus caritas\grqq!
Der zentrale Punkt ist mir aber, dass hinter diesen Diskussionen spürbar ist, dass es Cartellbrüder gibt, denen unser Verband und unsere Verbindungen wichtig sind. Heute ist das für mich ein stärkender Impuls entgegen der weitverbreiteten Egozentrik und Selbstoptimierung in unserer Gesellschaft.
Ich würde mich freuen, wenn wir auch im Jahr unseres 120. Gründungs- und Stiftungsfests aktiv über die Weiterentwicklung unserer Vandalia sprechen und diskutieren. Was bedeuten unsere Prinzipien für uns auch im 21. Jahrhundert! Diese Auseinandersetzung in bundesbrüderlicher Freundschaft ist für ein “vivat, crescat, floreat at multos annos Vandalia” unbedingt notwendig! 

\end{multicols}
\begin{flushright}
		\hfill\emph{Dr. Michael Spickenreuther Va!}
	\end{flushright}