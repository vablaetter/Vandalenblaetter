\section{Vorwort des Seniors aus dem WS 2023/24}
\sectionmark{Vorwort des Seniors}

Liebe Freunde der Vandalia,\\
Liebe Bundesbrüder,
\begin{multicols}{2}

Es ist mir eine Freude, Dir die neuste Ausgabe der Vandalenblätter vorstellen zu dürfen. Dank gilt dabei vor allem den Redakteuren dieser Ausgabe, den Bundesbrüdern Philipp Keyzman und Benedikt Kirchhof, welche diese Ausgabe zusammengestellt haben und gegen viele Widrigkeiten die Prozesse, die hierfür notwendig sind, wiederbelebt oder gar neu erfunden haben. So hoffe ich, dass die Artikel Dir einen guten Einblick in das vergangene Wintersemester geben können und auch bei den Bundesbrüdern, welche daran aktiv teilgenommen haben, so manche schöne Erinnerung wieder hervorruft. An dieser Stelle möchte ich - auch stellvertretend für die Redaktur - Dich nochmal herzlich dazu einladen, gerne auch Deine eigenen Beiträge einzusenden. Ob Du aus Deiner akademischen Arbeit, Deinem Alltag, einer spannenden Reise berichten oder einfach einen Kommentar zum aktuellen Geschehen in der Verbindung und Welt verfassen möchtest: Gerne nehmen wir diese Artikel oder auch photographischen Interpretationen in die Blätter auf und fördern so den Austausch zwischen den Bundesbrüdern. Hierfür sind die Redakteure in Zukunft unter \textbf{vablaetter@vandalia.de} erreichbar, sodass die Artikel auch eine Amtsübergabe überdauern werden.

Zuletzt möchte ich Dir noch kurz die Chargen des vergangenen Wintersemesters 2023/24 vorstellen und anschließend einen kurzen Überblick über dieses geben. Als Kassier und Schriftführer konnten die Bundesbrüder Daniel Rampf und Raphael Frank erste Erfahrungen in der Chargenarbeit gewinnen. Sie wurden zusammen mit ihrem Leibbruder Elia Strasser auf dem Gründungsfestkommer geburscht. Die gesammelte Erfahrung prädestiniert sie dafür, in Zukunft auch höhere Verantwortung für unsere liebe Vandalia zu übernehmen. Als Fuxmajor kümmerte sich Bundesbruder Fabian Heinrich in seiner vierten Charge um die Keilarbeit sowie die Erziehung des Fuxenstalls. Der Hohe Consenior, Bundesbruder Derek Dominguez, kam im WS 2022/23 als ZM zu unserer Vandalia. Zuvor besuchte er in Hamburg das Studienkolleg und wurde im WS 2021/22 bei e.V. K.D.St.V. Wiking Hamburg rezipiert. Er bereicherte das Semester und insbesondere die Begrüßungsabende durch Gerichte aus seiner ecuadorianischen Heimat. Als Senior durfte ich, Konrad Schönleber, wie bereits im Sommersemester unserer lieben Vandalia vorstehen.
\\
Das Semester stand unter dem Wahlspruch \textit{"Wo sich Männr finden, die für Ehr und Recht, mutig sich verbinden, weilt ein frei Geschlecht"}. Dieses Zitat aus dem Gedicht "Freiheit, die ich meine" von dem Dichter Max von Schenkendorf wurde etwa in einigen Kneipreden erörtert. Neben diesen obligatorischen Kneipen fand im Semester auch eine Rezeptionskneipe statt. Es wurden auf jenen Veranstaltungen sieben Füxe rezipiert, von welchen uns leider im Laufe des Semesters zwei verlassen mussten, sodass sich der Fuxenstall mit nun fünf Füxen doch adäquat gefüllt zeigt. Höhepunkt des Semesters war das vom 24. bis 26. November stattfindende 119. Gründungsfest. Dieses war herausragend besucht und wir hatten die Ehre, den 1. Vizepräsidenten aD des Bayrischen Landtags, Karl Freller hierzu als Festredner begrüßen zu dürfen. Das Grundprinzip der religio war unter anderem durch Besuche der Roratemessen oder auch dem Pater-Rupert-Mayer-Gedächtnisgottesdienst vertreten. Auch die scientia kam mit verschiedenen Vorträgen und Museumsbesuchen nicht zu kurz. Neben der Feuerzangenbowle fand auch eine familiäre Adventsfeier als eine winterliches Äquivalent des Maifests statt. Hierzu wurde zwar eine größere Menge an Gebäck gebacken, allerdings konnte sie wegen Krankheit vieler Bundesbrüder nur in recht kleinem Kreise stattfinden. Man könnte sie vermutlich in den kommenden Jahren mit wenig Aufwand zu einer größeren Veranstaltung ausbauen, wenn man dies den Bundesbrüdern besser kommunizieren würde. Anfang November gab es zudem schon ein frühes, kleines Feierwochenende: Bei einem zweitägigen Tanzkurs wurden die Kenntnisse der Bundesbrüder für die kommende Ballsaison aufgefrischt. Zudem durften wir uns an diesem Samstag den den Genuss einer Martinsgans kommen. Hierfür gilt mein Herzlicher Dank an Fbr. Rampf e.V. K.St.V. Rechberg sowie an seine Frau für die sehr lehrreiche Durchführung des Tanzkurses und an unsere Bundesbrüder AH Michael Oppitz sowie AH Richard Gurtner, welche das Martinsgansessen organisiert und mit bester Kochkunst vorbereitet haben.
Doch zu den einzelnen Veranstaltungen kannst Du in den folgenden Artikeln detailliertere und hoffentlich weniger langweilige Worte finden. So bleibt mir nur, meinen Dank an alle Bundesbrüder und insbesondere an die Philisterchargen für die Unterstützung vor und während des Semesters auszusprechen. Ich bin höchst erfreut, dass ich unserer Vandalia zwei Semester vorstehen durfte und hier in der schönsten Stadt am Isarstrande eine zweite Heimat in unserer Verbindung gewinnen konnte.  

Auf dass diese Freude noch vielen weiteren Bundesbrüdern zuteil werden wird:

Vivat, crescat, floreat Vandalia ad multos annos!
	%
	\begin{flushright}
		\hfill\emph{Konrad Schönleber RBo! Va!}
	\end{flushright}
	%
\end{multicols}

